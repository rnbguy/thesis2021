%!TEX root = ../../Thesis.tex

\section{Intractability for Registers, Sets, Flags, and Counters}
\label{sec:intractability}

In this section we demonstrate that checking the consistency is intractable for many widely-used data types. While this is not completely unexpected, since some related consistency-checking problems like sequential consistency are also intractable~\cite{DBLP:journals/siamcomp/GibbonsK97}, this contrasts recent tractability results for checking strong consistency (i.e.,~linearizability) of common non-replicated data types like sets, maps, and queues~\cite{DBLP:journals/pacmpl/EmmiE19}. In fact, in many cases we show that intractability even holds if the number of replicas is fixed.

Our proofs of intractability follow the general structure of Gibbons and Korach’s proofs for the intractability of checking sequential consistency (SC) for atomic registers with read and write operations~\cite{DBLP:journals/siamcomp/GibbonsK97}. In particular, we reduce a specialized type of NP-hard propositional satisfiability (SAT) problem to checking whether histories are admitted by a given data type. While our construction borrows from Gibbons and Korach’s, the adaptation from SC to CRDT consistency requires a significant extension to handle the consistency relaxation represented by abstract executions: rather than a direct sequencing of threads’ operations, CRDT consistency requires the construction of three separate relations: read-from, happens-before, and linearization.

Technically, our reductions start from the 1-in-3 SAT problem~\cite{DBLP:books/fm/GareyJ79}: given a propositional formula $\bigwedge_{i=1}^{m} (\alpha_i \lor \beta_i \lor \gamma_i)$ over variables $x_1, \ldots, x_n$ with only positive literals, i.e.,~$\alpha_i, \beta_i, \gamma_i \in \set{ x_1, \ldots, x_n }$, does there exist an assignment to the variables such that exactly one of $\alpha_i, \beta_i, \gamma_i$ per clause is assigned $\mathit{true}$? The proofs of Theorems~\ref{thm:3sat-to-flags} and~\ref{thm:3sat-to-counter} reduce 1-in-3 SAT to CRDT consistency checking.

\vspace{-1mm}
\begin{theorem}
  \label{thm:3sat-to-flags}

  The admissibility problem is NP-hard when the number of replicas is fixed for the following data types: Add-Wins Set, Remove-Wins Set, Enable-Wins Flag, Disable-Wins Flag, Multi-Value Register, and Last-Writer-Wins Register.
\vspace{-1mm}
\end{theorem}

\begin{proof}

  We demonstrate a reduction from the 1-in-3 SAT problem. For a given problem $p = \bigwedge_{i=1}^{m} (\alpha_i \lor \beta_i \lor \gamma_i)$ over variables $x_1, \ldots, x_n$, we construct a 3-replica history $h_p$ of the flag data type — either enable- or disable-wins — as illustrated in Figure~\ref{fig:3sat-to-flags}. The encoding includes a flag variable~$x_j$ for each propositional variable $x_j$, along with a per-replica flag variable $y_j$ used to implement synchronization barriers. Intuitively, executions of $h_p$ proceed in $m+1$ rounds: the first round corresponds to the assignment of a truth valuation, while subsequent rounds check the validity of each clause given the assignment. The reductions to sets and registers are slight variations on this proof, in which the \textrm{Read}, \textrm{Enable}, and \textrm{Disable} operations are replaced with \textrm{Contains}, \textrm{Add}, and \textrm{Remove}, respectively, and \textrm{Read} and \textrm{Writes} of values 1 and 0, respectively. Now it suffices to show that the constructed history $h_p$ is admitted if and only if the given problem $p$ is satisfiable.
  
  \begin{lemma}
    \label{crdt:flag:npc-proof:lemma3}
    $p = \bigwedge_{i=1}^{m} (\alpha_i \lor \beta_i \lor \gamma_i)$ is satisfied if and only if $h_p$ is admissible
  \end{lemma}
  
  Since the flag data type does not constrain the linearization relation of its abstract executions, we regard only the read-from and happens-before components. The construction of $h_p$ ensures the happens-before relations of its abstract executions:
  \vspace{-1.5mm}
  \begin{enumerate}

    \item do not interleave operations from different rounds. Each consecutive rounds are separated by the barriers in happens-before relations; and

    \item at each round, only one replica, say replica $i$, can finish its \textrm{Read}s then finish its \textrm{Enable}s/\textrm{Disable}s, then $(i+1) \bmod 3$ replica can finish its \textrm{Read}s and so on. And these \textrm{Enable}s/\textrm{Disable}s from one round are totally ordered between replicas by the happens-before relation.

\vspace{-1.5mm}
  \end{enumerate}

  In other words, replicas appear to execute atomically per round, in a round-robin fashion. Furthermore, since all operations in a given round happen before the operations of subsequent rounds, the values of flag variables are consistent across rounds — i.e.,~as read by the first replica to execute in a given round — and determined in the initial round either by conflict resolution — i.e.,~enable- or disable-wins — or by happens-before, in case conflict resolution would have been inconsistent with subsequent reads.

  \begin{figure}[t]
    \centering
    {\scriptsize\setlength{\tabcolsep}{2em}
\begin{tabular}{rlll}

  & Replica 0 & Replica 1 & Replica 2 \\
  \cmidrule(lr){2-2}
  \cmidrule(lr){3-3}
  \cmidrule(lr){4-4}

  \ldelim\{{3}{2em}[Round 0]
  & $\mathrm{Enable}(x_1)$  & $\mathrm{Disable}(x_1)$ & \\
  & \ldots                  & \ldots                  & \\
  & $\mathrm{Enable}(x_n)$  & $\mathrm{Disable}(x_n)$ & \\[1em]

  \ldelim\{{3}{2em}[Barrier 1]
  & $\mathrm{Enable}(y_0)$      & $\mathrm{Enable}(y_1)$      & $\mathrm{Enable}(y_2)$      \\
  & $\mathrm{Read}(y_1) = \mathit{true}$ & $\mathrm{Read}(y_0) = \mathit{true}$ & $\mathrm{Read}(y_0) = \mathit{true}$ \\
  & $\mathrm{Read}(y_2) = \mathit{true}$ & $\mathrm{Read}(y_2) = \mathit{true}$ & $\mathrm{Read}(y_1) = \mathit{true}$ \\[1em]

  \ldelim\{{5}{2em}[Round 1]
  & $\mathrm{Read}(\alpha_1) = \mathit{true}$  & $\mathrm{Read}(\alpha_1) = \mathit{false}$  & $\mathrm{Read}(\alpha_1) = \mathit{false}$  \\
  & $\mathrm{Read}(\beta_1) = \mathit{false}$   & $\mathrm{Read}(\beta_1) = \mathit{true}$   & $\mathrm{Read}(\beta_1) = \mathit{false}$   \\
  & $\mathrm{Read}(\gamma_1) = \mathit{false}$  & $\mathrm{Read}(\gamma_1) = \mathit{false}$  & $\mathrm{Read}(\gamma_1) = \mathit{true}$  \\
  & $\mathrm{Disable}(\alpha_1)$      & $\mathrm{Disable}(\beta_1)$       & $\mathrm{Disable}(\gamma_1)$      \\
  & $\mathrm{Enable}(\beta_1)$        & $\mathrm{Enable}(\gamma_1)$       & $\mathrm{Enable}(\alpha_1)$       \\[1em]

  \ldelim\{{3}{2em}[Barrier 2]
  & $\mathrm{Disable}(y_0)$     & $\mathrm{Disable}(y_1)$     & $\mathrm{Disable}(y_2)$      \\
  & $\mathrm{Read}(y_1) = \mathit{false}$ & $\mathrm{Read}(y_0) = \mathit{false}$ & $\mathrm{Read}(y_0) = \mathit{false}$ \\
  & $\mathrm{Read}(y_2) = \mathit{false}$ & $\mathrm{Read}(y_2) = \mathit{false}$ & $\mathrm{Read}(y_1) = \mathit{false}$ \\[1em]

  & \ldots & \ldots & \ldots \\[1em]

  \ldelim\{{5}{2em}[Round m]
  & $\mathrm{Read}(\alpha_m) = \mathit{true}$  & $\mathrm{Read}(\alpha_m) = \mathit{false}$  & $\mathrm{Read}(\alpha_m) = \mathit{false}$  \\
  & $\mathrm{Read}(\beta_m) = \mathit{false}$   & $\mathrm{Read}(\beta_m) = \mathit{true}$   & $\mathrm{Read}(\beta_m) = \mathit{false}$   \\
  & $\mathrm{Read}(\gamma_m) = \mathit{false}$  & $\mathrm{Read}(\gamma_m) = \mathit{false}$  & $\mathrm{Read}(\gamma_m) = \mathit{true}$  \\
  & $\mathrm{Disable}(\alpha_m)$      & $\mathrm{Disable}(\beta_m)$       & $\mathrm{Disable}(\gamma_m)$      \\
  & $\mathrm{Enable}(\beta_m)$        & $\mathrm{Enable}(\gamma_m)$       & $\mathrm{Enable}(\alpha_m)$       \\

\end{tabular}
}
         \vspace{-1mm}
    \caption{The encoding of a 1-in-3 SAT problem $\bigwedge_{i=1}^{m} (\alpha_i \lor \beta_i \lor \gamma_i)$ over variables $x_1, \ldots, x_n$ as a 3-replica history of a flag data type. Besides the flag variable $x_j$ for each propositional variable $x_j$, the encoding adds per-replica variables $y_j$ for synchronization barriers.}
    \label{fig:3sat-to-flags}
    \vspace{-4mm}
  \end{figure}


  \begin{proof}
    Only-if direction.
    When $\bigwedge_{i=1}^{1} (\alpha_i \lor \beta_i \lor \gamma_i)$ is satisfied, that means, there exists an assignment for which all the first clause have exactly one literal set to true. First for all, to assign the corresponding values to $x_i$, we construct the $\hb$ such way that if $x_i = false$ in the SAT formula, then we make $Enable(x_i)$ in Replica 0 visible to $Disable(x_i)$ in Replica 1, \ie $(Enable(x_i), Disable(x_i)) \in \hb$. This is make sure value of $x_i$ is $false$ after Barrier 1. Similarly we $(Disable(x_i), Enable(x_i)) \in \hb$ if $x_i = true$ in SAT formula. Note this does not introduce any cycle in $\hb$ because $x_i$s are \textrm{Enable}d and \textrm{Disable}d in same order in Replica 0 and Replica 1.

    Now at barrier i, we add all the \textrm{Enable} happens-before all the \textrm{Read} in $\hb$. This also makes sure there is no cycle.

    Now for each round i, if $\alpha_i$ is true in clause i, then we make the round i of replica 0 happen-before the round i of replica 1 in $\hb$, then the round i of replica 1 happens-before the round i of replica 2 in $\hb$. This makes the history admissible because, at first $\alpha_1$ is true and $\beta_1$ and $\gamma_1$ are false. So the reads of round i in replica 0 goes ahead. Then the updates of round i in replica 0 make $\alpha_1$ false and $\beta_1$ true. $\gamma_1$ stays false. So now, the reads of round i in replica 1 can go ahead. Similarly the happens before relation works between replica 1 and replica 2. So the sub-history till round i is admissible.
    
    If $\beta_i$ is true then the happens-order between replicas are is 1, 2 then 0 and $\gamma_1$ is true, it is 2, 0 and then 1.

    So we have a happens-before relation which admits each round and barrier, so the history is admissible.

    Before we prove if-direction, we prove some lemmas for the properties of admissible history.

  \begin{lemma}
    \label{crdt:flag:npc-proof:lemma1}
    The \textrm{Read}s of each barrier reads from the \textrm{Enable}s or \textrm{Disable}s of the same barrier. 
  \end{lemma}

  \begin{proof}
    We prove by induction on the number of each barrier.
    Base case. $k = 1$. Note that only replica $i$ \textrm{Enable}s or \textrm{Disable}s $y_i$. At barrier 1, if the $\mathrm{Read}(y_j) = true$ at replica $i$ is read from a barrier other than 1, then it must have read from barrier 3 at least. Because $y_j$ is set to true at odd numbered barriers. Now at barrier 2 and replica j, $\mathrm{Read}(y_i) = false$ reads from a \textrm{Disable} because replica j has $\mathrm{Read}(y_i) = true$ at barrier 1. So at barrier 2 and replica j, $\mathrm{Read}(y_i) = false$ can read from a \textrm{Disable} which is at even numbered barriers, atleast from barrier 2.
    
    So we found a cycle in happens-before order. $\mathrm{Enable}(y_j)$ at barrier $\geq3$ and replica $j$ happens-before $\mathrm{Read}(y_j) = true$ at barrier 1 and replica $i$ because of read-from. Then $\mathrm{Read}(y_j) = true$ at barrier 1 and replica $i$ happens-before $\mathrm{Disable}(y_i)$ at barrier $\geq2$ and replica $i$. Then $\mathrm{Disable}(y_i)$ at barrier $\geq2$ and replica $i$ happens-before $\mathrm{Read}(y_i) = false$ at barrier 2 and replica $j$ because of read-form. Then$\mathrm{Read}(y_i) = false$ at barrier 2 and replica $j$ happens-before $\mathrm{Enable}(y_j)$ at barrier $\geq3$ and replica $j$.
    
    Inductive step. By induction hypothesis, barrier $k$ always reads from barrier $k$ itself. Now at $\mathrm{Read}(y_j)$ of barrier $(k+1)$, the history is visible upto the write of $y_j$ from barrier $k$. Without loss of generality, let's assume, barrier $k$ writes $false$ value. Since barrier $k$ writes $true$, the $false$ reads of barrier $(k+1)$ must read from a barrier strictly greater than $k$. Using the same logic from base case, we can find a cycle in the happens-before relation. The argument stays the similar when barrier $k$ writes $true$ value.
  \end{proof}

Lemma \ref{crdt:flag:npc-proof:lemma1} ensures, all the operations from all replicas before barrier i are visible to each replica after barrier i.

  Two \textrm{Read} operations \emph{see the same value of $x_i$} when one reads-from a resolved $\mathrm{Enable}(x_i)$(or $\mathrm{Disable}(x_i)$) if and only if the other one also reads-from a resolved $\mathrm{Enable}(x_i)$(or $\mathrm{Disable}(x_i)$). But these two \textrm{Enable}(or \textrm{Disable}) operations may not be the same.

  \begin{lemma}
    \label{crdt:flag:npc-proof:lemma2}
    $\hb$-minimal \textrm{Read}s of each round$_{\geq1}$ sees the same values of $x_i$.
  \end{lemma}

  \begin{proof}
    We prove by induction on the number of rounds. But there is no base case. We will just show that at the round $(k+1)$ the maximally first \textrm{Read}s sees the the same assignments as the maximally first \textrm{Read}s of round $k$.

    We will show at each round, the operations from each replica happen in a total order, that too one replica after another. Explicitly, for some some replica $i$, its operations happen first, then the operations from replica $(i+1) \bmod 2$ happen, then the operations from replica $(i+2) \bmod 2$ happen. 

    First of all, any the $\hb$-minimal \textrm{Read} of each round sees exactly the same resolved values of $x_i$ till last round because of lemma \ref{crdt:flag:npc-proof:lemma1}. Also, the \textrm{Read}s from each round are only reading from updates from the same round or any before round. Reading from any later round is not possible, because that will introduce a cycle between current round and that later round in $\hb$ because of lemma \ref{crdt:flag:npc-proof:lemma1}.

    Also, in each round, there exist one replica which does not read from other replicas in the same round. If replica $p$ is reading from replica $q$, then replica $q$ again can not read from replica $p$, cause it will create a cycle in $\hb$ between replica $p$ and $q$ in the same round. So replica $q$ has to read from replica $r$. But then replica $r$ will have to read from replica $p$, which creates a cycle between replica $p$, $q$ and $r$ in the same round.

    So there exists one replica, which reads the resolved values till last round. Since the \textrm{Read}s of that are successful, it ensures only one of $\alpha_k, \beta_k, \gamma_k$ are true. Hence, the first true \textrm{Read}s at other replicas must read-from the updates at the same round. That totally orders the operations of one round. That is, if replica 0 was the first one to finish its reads, $\mathrm{Read}(\beta_k) = true$ read-from $\mathrm{Enable}(\beta_k)$ and $\mathrm{Read}(\gamma_k) = true$ read-from $\mathrm{Enable}(\gamma_k)$. Since the updates are totally ordered and they only flips the read values of $x_i$ twice, \ie if the first read on $x_i$ is $false$, then the resolved update on $x_i$ till round $(k-1)$ was \textrm{Disable} and at round $k$, after $\mathrm{Read}(x_i) = false$, $\mathrm{Enable}(x_i)$ and $\mathrm{Disable}(x_i)$ happen and they are ordered in $\hb$. The same argument holds when the first read on $x_i$ was $true$. 

    When round $(k+1)$ begins, because of lemma \ref{crdt:flag:npc-proof:lemma1}, it sees the resolved updates at the end of round $k$. So at $\hb$-minimal \textrm{Read}s of round $k$, if the resolved update on $x_i$ was \textrm{Enable}(or \textrm{Disable}), at $\hb$-minimal \textrm{Read}s of round $(k+1)$, the resolved update for $x_i$ stays \textrm{Enable}(or \textrm{Disable}) also.
  \end{proof}

  Now we go back to our main lemma \ref{crdt:flag:npc-proof:lemma3}. We use the same assignment of $x_i$s at the first \textrm{Read}s of each round (they are the same by lemma \ref{crdt:flag:npc-proof:lemma2}) for the SAT formula.

  Now we show that assignment satisfies the 1-in-3 SAT formula. Suppose, there is a clause $i$ which does not have exactly one positive literal. If we look at round $i$, there is a replica which happens-before the other replicas. The \textrm{Read}s of that replica sees the same $x_i$ assignment as the SAT assignment, yet they see exactly one true value. Which contradicts our assumption that the assignment makes clause $i$ unsatisfied.
\end{proof}

We proved that this reduction from a NP-complete problem to our admissibility works. It is very easy to the size of the history is linear in the size of formula and can be computed in linear-time. Theorem \ref{thm:3sat-to-flags} is proved.

%   In the “if” direction, let $\vec{r} \in \set{0,1,2}^m$ be the positions of positively-assigned variables in each clause, e.g.,~$r_i = 0$ implies $\alpha_i = \mathit{true}$ and $\beta_i = \gamma_i = \mathit{false}$. We construct an abstract execution $e_{\vec{r}}$ in which the happens-before relation sequences the operations of replica~$r_i$ before those of $r_i + 1 \bmod 3$, and in turn before $r_i + 2 \bmod 3$. In other words, the replicas in round $i$ appear to execute in left-to-right order from starting with the replica~$r_i$, whose reads correspond to the satisfying assignment of $(\alpha_i \lor \beta_i \lor \gamma_i)$. The read-from relation of $e_{\vec{r}}$ relates each $\mathrm{Read}(x_j) = \mathit{true}$ operation to the most recent $\mathrm{Enable}(x_j)$ operation in happens-before order, which is unique since happens-before sequences the operations of all rounds; the case for $\mathrm{Read}(x_j) = \mathit{false}$ and $\mathrm{Disable}(x_j)$ is symmetric. It is then straightforward to verify that $e_{\vec{r}}$ satisfies the axioms of the enable- or disable-wins flag, and thus $h_p$ is admitted.

%   In the “only if” direction, let $e$ be an abstract execution of $h_p$, and let $\vec{r} \in \set{0,1,2}^m$ be the replicas first to execute in each round according to the happens-before order of $e$. It is straightforward to verify that the assignment in which a given variable is set to true if{f} the replica encoding its positive assignment in some clause executes first in its round, i.e.,
% \vspace{-1mm}
%   \begin{align*}
%     x_j = \left\{
%     \begin{array}{ll}
%       \mathit{true} \qquad & \text{ if } \exists i. (r_i = 0 \land \alpha_i = x_j) \lor (r_i = 1 \land \beta_i = x_j) \lor (r_i = 2 \land \gamma_i = x_j) \\
%       \mathit{false} & \text{ otherwise,}
%     \end{array}
%     \right. \\[-7mm]
%   \end{align*}
%   is a satisfying assignment to $p$.
% \vspace{-1mm}
\end{proof}

Theorem~\ref{thm:3sat-to-flags} establishes intractability of consistency for the aforementioned sets, flags, and registers, independently from the number of replicas. In contrast, our proof of Theorem~\ref{thm:3sat-to-counter} for counter data types depends on the number of replicas, since our encoding requires two replicas per propositional variable. Intuitively, since counter increments and decrements are commutative, the initial round in the previous encoding would have fixed all counter values to zero. Instead, the next encoding isolates initial increments and decrements to independent replicas.
% Todo:
The weaker result is indeed tight since checking counter consistency with a fixed number of replicas is polynomial time, as Section~\ref{sec:counter} demonstrates.

\vspace{-1mm}
\begin{theorem}
  \label{thm:3sat-to-counter}

  The admissibility problem for the Counter data type is NP-hard.

\vspace{-1mm}
\end{theorem}

We demonstrate a reduction from the 1-in-3 SAT problem. For a given problem $p = \bigwedge_{i=1}^{m} (\alpha_i \lor \beta_i \lor \gamma_i)$ over variables $x_1, \ldots, x_n$, we construct a history $h_p$ of the counter data type over $2n+3$ replicas, as illustrated in Figure~\ref{fig:3sat-to-counter}.

    \begin{figure}[t]
      \centering
      {\scriptsize\setlength{\tabcolsep}{0.8em}
\begin{tabular}{rlll}

  & Replica 0 & Replica $2j\!+\!1$ & Replica $2j\!+\!2$  \\
  \cmidrule(lr){2-2}
  \cmidrule(lr){3-3}
  \cmidrule(lr){4-4}

  \ldelim\{{3}{4em}[Round 0]
  & & $\mathrm{Inc}(y)$   & $\mathrm{Inc}(y)$ \\
  & & $\mathrm{Inc}(x_j)$ & $\mathrm{Dec}(x_j)$ \\

  & $\mathrm{Read}(y) = n$ \\[1em]

  & & Replica 1 & Replica 2 \\
  \cmidrule(lr){3-3}
  \cmidrule(lr){4-4}

  \ldelim\{{3}{4em}[Barrier 1]
  & $\mathrm{Inc}(y_0)$      & $\mathrm{Inc}(y_1)$      & $\mathrm{Inc}(y_2)$ \\
  & $\mathrm{Read}(y_1) = 1$ & $\mathrm{Read}(y_2) = 1$ & $\mathrm{Read}(y_0) = 1$ \\
  & $\mathrm{Read}(y_2) = 1$ & $\mathrm{Read}(y_0) = 1$ & $\mathrm{Read}(y_1) = 1$ \\[1em]

  \ldelim\{{5}{4em}[Round 1]
  & $\mathrm{Read}(\alpha_1) = 1$   & $\mathrm{Read}(\beta_1) = 1$  & $\mathrm{Read}(\gamma_1) = 1$  \\
  & $\mathrm{Read}(\beta_1) = -1$   & $\mathrm{Read}(\gamma_1) = -1$    & $\mathrm{Read}(\alpha_1) = -1$   \\
  & $\mathrm{Read}(\gamma_1) = -1$  & $\mathrm{Read}(\alpha_1) = -1$  & $\mathrm{Read}(\beta_1) = -1$   \\
  & $\mathrm{Dec}(\alpha_1)$; $\mathrm{Dec}(\alpha_1)$ & $\mathrm{Dec}(\beta_1)$; $\mathrm{Dec}(\beta_1)$ & $\mathrm{Dec}(\gamma_1)$; $\mathrm{Dec}(\gamma_1)$ \\
  & $\mathrm{Inc}(\beta_1)$; $\mathrm{Inc}(\beta_1)$ & $\mathrm{Inc}(\gamma_1)$; $\mathrm{Inc}(\gamma_1)$ & $\mathrm{Inc}(\alpha_1)$; $\mathrm{Inc}(\alpha_1)$ \\[1em]

  \ldelim\{{3}{4em}[Barrier 2]
  & $\mathrm{Dec}(y_0)$      & $\mathrm{Dec}(y_1)$      & $\mathrm{Dec}(y_2)$ \\
  & $\mathrm{Read}(y_1) = 0$ & $\mathrm{Read}(y_2) = 0$ & $\mathrm{Read}(y_0) = 0$ \\
  & $\mathrm{Read}(y_2) = 0$ & $\mathrm{Read}(y_0) = 0$ & $\mathrm{Read}(y_1) = 0$ \\[1em]

  & \ldots & \ldots & \ldots \\[1em]

  \ldelim\{{5}{4em}[Round $m$]
  & $\mathrm{Read}(\alpha_m) = 1$   & $\mathrm{Read}(\beta_m) = 1$  & $\mathrm{Read}(\gamma_m) = 1$  \\
  & $\mathrm{Read}(\beta_m) = -1$   & $\mathrm{Read}(\gamma_m) = -1$    & $\mathrm{Read}(\alpha_m) = -1$   \\
  & $\mathrm{Read}(\gamma_m) = -1$  & $\mathrm{Read}(\alpha_m) = -1$  & $\mathrm{Read}(\beta_m) = -1$   \\
  & $\mathrm{Dec}(\alpha_m)$; $\mathrm{Dec}(\alpha_m)$ & $\mathrm{Dec}(\beta_m)$; $\mathrm{Dec}(\beta_m)$ & $\mathrm{Dec}(\gamma_m)$; $\mathrm{Dec}(\gamma_m)$ \\
  & $\mathrm{Inc}(\beta_m)$; $\mathrm{Inc}(\beta_m)$ & $\mathrm{Inc}(\gamma_m)$; $\mathrm{Inc}(\gamma_m)$ & $\mathrm{Inc}(\alpha_m)$; $\mathrm{Inc}(\alpha_m)$ \\[1em]

  \ldelim\{{3}{6em}[Barrier $m\!+\!1$]
  & $\mathrm{Inc}(y_0)$ or $\mathrm{Dec}(y_0)$  & $\mathrm{Inc}(y_1)$ or $\mathrm{Dec}(y_1)$     & $\mathrm{Inc}(y_2)$ or $\mathrm{Dec}(y_2)$      \\
  & $\mathrm{Read}(y_1) = 1 \text{ or } 0$ & $\mathrm{Read}(y_2) = 1 \text{ or } 0$ & $\mathrm{Read}(y_0) = 1 \text{ or } 0$ \\
  & $\mathrm{Read}(y_2) = 1 \text{ or } 0$ & $\mathrm{Read}(y_0) = 1 \text{ or } 0$ & $\mathrm{Read}(y_1) = 1 \text{ or } 0$ \\[1em]

  \ldelim\{{1}{5em}[Round $m\!+\!1$]
  & $\mathrm{Read}(y) = n$ \\

\end{tabular}
}
     \vspace{-1mm}
      \caption{The encoding of a 1-in-3 SAT problem $\bigwedge_{i=1}^{m} (\alpha_i \lor \beta_i \lor \gamma_i)$ over variables $x_1, \ldots, x_n$ as the history of a counter over $2n+3$ replicas. Besides the counter variables $x_j$ encoding propositional variables $x_j$, the encoding adds a variable $y$ encoding the number of initial increments and decrements, and a variable $z$ to implement synchronization barriers.}
      \label{fig:3sat-to-counter}
     \vspace{-4mm}
    \end{figure}

     Besides the differences imposed due to the commutativity of counter increments and decrements, our reduction follows the same strategy as in the proof of Theorem~\ref{thm:3sat-to-flags}: the happens-before relation of $h_p$’s abstract executions order every pair of operations in distinct rounds (of Replicas 0–2), and every operation in a given (non-initial) round. As before, Replicas 0–2 appear to execute atomically per round, in a round-robin fashion, and counter variables are consistent across rounds. The key difference is that here abstract executions’ happens-before relations only relate the operations of either Replica~$2j\!+\!1$ or $2j\!+\!2$, for each $j = 1, \ldots, n$, to operations in subsequent rounds: the other’s operations are never observed by other replicas. Our encoding ensures that exactly one of each is observed by ensuring that the counter~$y$ is incremented exactly $n$ times — and relying on the fact that every variable appears in some clause, so that a read that observed neither or both would yield the value zero, which is inconsistent with $h_p$. Otherwise, our reasoning follows the proof of Theorem~\ref{thm:3sat-to-flags}, in which the read-from relation selects all increments and decrements of the same counter variable in happens-before order.
\vspace{-2mm}

% \begin{lemma}
%   \label{crdt:counter:npc-proof:lemma1}
%   The \textrm{Read}s of each barrier reads from the \textrm{Inc}s or \textrm{Dec}s of the same barrier. 
% \end{lemma}

% \begin{proof}
%   Base case. At barrier 1, let's assume replica $i$'s $\mathrm{Read}(z) = 3$ happened-before at first. Then, it must have been three Inc(z) before it. Since the $\mathrm{Read}(z) = 3$ are not performed, the only $Inc(z)$ are available from same barrier but from other replicas. If the second $\mathrm{Read}(z) = 3$ reads from other barrier, then the first replica progressed through some other barrier which $\mathrm{Read}(z) = 0$, which is not possible. So the second read also reads from barrier 1. Similarly the third read also read from barrier 1.

%   Inductive step. This step is very similar to the induction step of lemma \ref{crdt:flag:npc-proof:lemma1}. Once we prove the base case, we show that the possible reads can only happen from barriers after the current barrier. Then just like the base, we prove reading from any barrier after the current one is not possible. 
% \end{proof}

% \begin{lemma}
%   \label{crdt:counter:npc-proof:lemma2}
%   At the completion of each sub-history till each round has same assignment for $x_i$s.
% \end{lemma}

% \begin{proof}
%   The proof of lemma \ref{crdt:counter:npc-proof:lemma2} and lemma \ref{crdt:counter:npc-proof:lemma3} are very similar to the proof of lemma \ref{crdt:flag:npc-proof:lemma2} and \ref{crdt:flag:npc-proof:lemma3}. Instead of $\mathrm{Enable}$, $\mathrm{Distable}$, $\mathrm{Read}=true$ and $\mathrm{Read}=false$, we have two $\mathrm{Inc}$s, two $\mathrm{Dec}$s, $\mathrm{Read}=1$, $\mathrm{Read}=-1$. Otherwise the proof stays the same.
% \end{proof}

% \begin{lemma}
%   \label{crdt:counter:npc-proof:lemma3}
%   $p = \bigwedge_{i=1}^{j \leq m} (\alpha_i \lor \beta_i \lor \gamma_i)$ is satisfied if and only if the sub-history till round j is admissible.
% \end{lemma}

% \begin{proof}
%   Theorem \ref{thm:3sat-to-counter} is a direct consequence of lemma \ref{crdt:counter:npc-proof:lemma3} when $j = m$.
% \end{proof}
