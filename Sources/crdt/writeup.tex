%!TEX root = Thesis.tex

\section{Old formalization}

We are trying to verify conflict-free data type or CRDT based operations.

Given a history of a CRDT, we want to verify if there is a valid execution of the CRDT replicas which generates the history.

\begin{definition}
 A history $\hist$ is $\tup{\Op, \RO}$, a tuple of operations set $\Op$ and the replica order $\RO$.
\end{definition}

% Each operation is of the form $\tup{\op, \id}$ where $\op$ is a record of an operation call on a CRDT replica and $\id$ is the identifier of the operation.

Each operation $\op \in \Op$ contains the record of an operation call on a CRDT replica.

$\RO$ is the replica orders. It totally orders the operations on a same replica. That is, if $\op_1$ and $\op_2$ are on same replicas then either $\tup{\op_1, \op_2} \in \RO$ or $\tup{\op_2, \op_2} \in \RO$. Also if $\op_1$ and $\op_2$ are operations on two different replicas, then $\tup{\op_1, \op_2}$, $\tup{\op_2, \op_2} \not\in \RO$.

\begin{definition}
 Happens before relation, $\hb$, relates two operations $\op_1$ and $\op_2$ such that $\op_1$ happens before $\op_2$ or $\op_1$ is visible to $\op_2$.
\end{definition}


\begin{definition}
 $\rf$ relates an update operation $\op_1$ to a corresponding read operation $\op_2$. It means there is no other $\op_3$ such that $\tup{\op_1, \op_3} \in \hb$ and $\tup{\op_3, \op_2} \in \hb$ such that $\op_3$ can affect the update performed by $\op_1$. Intuitively, $\rf$ relates the maximal concurrent updates to a read.
\end{definition}

$\hb$ must be acyclic \ie $\hb$ is a partial order.

$(\rf \cup \RO)^+ \subseteq \hb$

Now we define two axioms for casual consistent CRDT history.

A history $\tup{\Op, \RO}$ is a causal consistent CRDT history if $\exists \rf, \hb$ such that,

\begin{itemize}
 \item If $\op_1$ and $\op_2$ are in 'write-conflict' and $\op_3$ is in 'read-conflict' with $\op_1, \op_2$ for the same variable $x$ and $\tup{\op_1, \op_2}$, $\tup{\op_2, \op_3} \in \hb$. Then $\tup{\op_1, \op_3} \not\in \rf$.

       Intuitively, $\op_1$ is not the latest update on $x$ to $\op_3$. Hence $\rf$ can not relate them.

 \item If $\{\op_1, \op_2, \cdots, \op_n\}$ are in 'write-conflict' with each other and $\op$ is in 'read-conflict' with each $\op_i$ for the same variable $x$.
       Then the read at $\op$ happened according to a 'resolution-rule' specific to a CRDT.

       Intuitively, $\op_i$ are all happened concurrently and are in write-conflict. So CRDT should somehow resolve the conflict.
\end{itemize}

Example, for multi value register,
\begin{figure}[ht]
  \centering
  \begin{minipage}{.45\textwidth}
    \input figures/mv-reg-eg-a
    \caption{$\tup{\op_1, \op_3} \not\in \rf$}
    \label{mvreg_eg:a}
  \end{minipage}
  \begin{minipage}{.45\textwidth}
    \input figures/mv-reg-eg-b
    \caption{$1, 2, 3 \in Rd(x)$ in $\op$}
    \label{mvreg_eg:b}
  \end{minipage}
\end{figure}


\section{Specific CRDT Cases}

\begin{figure}[ht]
  \centering
  \begin{minipage}{.45\textwidth}
    \input figures/lww-reg-a
    \caption{$\tup{\op_1, \op_3} \not\in \rf$}
    \label{lwwreg:a}
  \end{minipage}
  \begin{minipage}{.45\textwidth}
    \input figures/lww-reg-b
    \caption{If $\op$ reads $x$ value to be $2$, that means, there exists a global linearization $\lin \supseteq \hb$ of $\Op$ such $\tup{\op_1, \op_2}$, $\tup{\op_3, \op_2} \in \lin$}
    \label{lwwreg:b}
  \end{minipage}
  \caption{Last writer wins register}
\end{figure}

\begin{figure}[H]
  \centering
  \begin{minipage}{.45\textwidth}
    \input figures/mv-reg-a
    \caption{$\tup{\op_1, \op_3} \not\in \rf$}
    \label{mvreg:a}
  \end{minipage}
  \begin{minipage}{.45\textwidth}
    \input figures/mv-reg-b
    \caption{$Rd(x) = \{1,2,3\}$ in $\op$}
    \label{mvreg:b}
  \end{minipage}
  \caption{Multi value register}
\end{figure}

Counter:

If $\op_0$: $Rd(x)$ and $\hb^{-1}(\op_0) = \{\op \mid \tup{\op, \op_0} \in \hb\}$, then $Rd(x) = $ number of increment of $x$ in $\hb^{-1}(\op_0)$ - number of decrement of $x$ in $\hb^{-1}(\op_0)$.

This is because decrement and increment are commutative operations. So it does not matter in which order the increment and decrement happened on a variable, just whether the operation is visible affects the read operation on that variable.

Replicated growing array:

Because of its unique updates and removes, RGA has axioms for $\rf$.

\begin{itemize}
 \item $\op_1$: AddAfter(\_, a), $\op_2$: AddAfter(a, \_) $\Rightarrow$ $\tup{\op_1, \op_2} \in \rf$.
 \item $\op_1$: AddAfter(\_, a), $\op_2$: Remove(a, \_) $\Rightarrow$ $\tup{\op_1, \op_2} \in \rf$.
 \item $\op_1$: AddAfter(\_, a), $\op_2$: Read contains a $\Rightarrow$ $\tup{\op_1, \op_2} \in \rf$.
 \item $\op_1$: AddAfter(\_, a), $\op_2$: Read does not contain a $\Rightarrow$ $\tup{\op_1, \op_2} \in \hb$ implies $\exists$ $\op_3$: Remove(a) such that $\tup{\op_3, \op_2} \in \rf$.
\end{itemize}



\begin{figure}[ht]
  \centering
  \begin{minipage}{.45\textwidth}
    \input figures/rga-a
    \caption{Read in $\op_3$ contains c before b}
    \label{rga:a}
  \end{minipage}
  \begin{minipage}{.45\textwidth}
    \input figures/rga-b
    \caption{Without loss of generality, if Read in $\op$ contains b, c, d in this order, that implies, there exists a global linearization $\lin \supseteq \hb$ for which $\tup{\op_3, \op_2}$ , $\tup{\op_2, \op_1} \in \lin$ }
    \label{rga:b}
  \end{minipage}
  \caption{Replicated growing array}
\end{figure}

\begin{figure}[ht]
  \centering
  \begin{minipage}{.45\textwidth}
    \input figures/set-a1
    \caption{$\tup{\op_3, \op_1} \not\in \rf$}
    \label{set:a1}
  \end{minipage}
  \begin{minipage}{.45\textwidth}
    \input figures/set-a2
    \caption{$\tup{\op_3, \op_1} \not\in \rf$}
    \label{set:a2}
  \end{minipage}
  \begin{minipage}{.45\textwidth}
    \input figures/set-a3
    \caption{$\tup{\op_3, \op_1} \not\in \rf$}
    \label{set:a3}
  \end{minipage}
  \begin{minipage}{.45\textwidth}
    \input figures/set-a4
    \caption{$\tup{\op_3, \op_1} \not\in \rf$}
    \label{set:a4}
  \end{minipage}
  \begin{minipage}{.45\textwidth}
    \input figures/set-b1
    \caption{For Rem-wins-add set, Read in $\op$ will not contain $x$}
    \label{set:b1}
  \end{minipage}
  \begin{minipage}{.45\textwidth}
    \input figures/set-b2
    \caption{For Add-wins-rem set, Read in $\op$ will contain $x$}
    \label{set:b2}
  \end{minipage}
  \caption{CRDT Set}
\end{figure}



\begin{figure}[ht]
  \centering
  \begin{minipage}{.45\textwidth}
    \input figures/flag-a1
    \caption{$\tup{\op_3, \op_1} \not\in \rf$}
    \label{flag:a1}
  \end{minipage}
  \begin{minipage}{.45\textwidth}
    \input figures/flag-a2
    \caption{$\tup{\op_3, \op_1} \not\in \rf$}
    \label{flag:a2}
  \end{minipage}
  \begin{minipage}{.45\textwidth}
    \input figures/flag-a3
    \caption{$\tup{\op_3, \op_1} \not\in \rf$}
    \label{flag:a3}
  \end{minipage}
  \begin{minipage}{.45\textwidth}
    \input figures/flag-a4
    \caption{$\tup{\op_3, \op_1} \not\in \rf$}
    \label{flag:a4}
  \end{minipage}
  \begin{minipage}{.45\textwidth}
    \input figures/flag-b1
    \caption{For Disable-wins-enable set, Read in $\op$ will have $x$ disabled}
    \label{flag:b1}
  \end{minipage}
  \begin{minipage}{.45\textwidth}
    \input figures/flag-b2
    \caption{For Enable-wins-disable set, Read in $\op$ will have $x$ enabled}
    \label{flag:b2}
  \end{minipage}
  \caption{CRDT Flag}
\end{figure}


\section{Algorithm}

%!TEX root = Thesis.tex

% wip


\begin{algorithm}[H]
{\footnotesize
 \SetKwInOut{KwInput}{Input}
 \SetKwInOut{KwOutput}{Output}
 \KwIn{A LWWReg CRDT history $\hist = \tup{\Op, \ro}$}
 \KwOut{$\mathit{true}$ iff $\hist$ satisfies \textsc{Causal consistency}}
 \BlankLine
 \If{$\so\cup\wro$ is cyclic} {
  \Return{false}\;
 }
 $\co \leftarrow \so\cup\wro$\;
 \ForEach{$\xvar \in \vars{\hist}$}{
  \ForEach{$\tr_1 \neq \tr_2 \in T$ s.t. $\tr_1$ and $\tr_2$ write on $\xvar$}{
   \If{$\exists \tr_3.\ \tup{\tr_1,\tr_3}\in \wro[\xvar]\land \tup{\tr_2,\tr_3}\in (\so\cup\wro)^+$} { %\Path{\tr_2}{E_1^+}{\tr_3}, \Path{\tr_1}{\wro[\xvar]}{\tr_3}
    $\co \leftarrow \co \cup \{\tup{\tr_2, \tr_1}\}$\;
   }
  }
 }
 \eIf{$\co$ is cyclic}{
  \Return{false}\;
 }{
  \Return{true}\;
 }}
 \caption{Checking \textsc{Causal consistency}}
 \label{ccalgo:1}
\end{algorithm}

\begin{procedure}
  {\footnotesize
   \SetKwInOut{KwInput}{Input}
   \SetKwInOut{KwOutput}{Output}
   \KwIn{A LWWReg CRDT history $\hist = \tup{\Op, \ro}$}
   \KwOut{$\rf$}
   \BlankLine
   $\rf \leftarrow \emptyset$\;
   \ForEach{$\xvar \in \vars{\hist}$}{
    \ForEach{$\op_2 \in \Op$ s.t. $\op_2 = \tup{Read(x), a}$}{
    \eIf{$\exists \op_1 \in \Op$ s.t. $\op_1 = \tup{Write(x, a), \bot}$}{
      $\rf \leftarrow \rf \cup \{\tup{\op_1, \op_2}\}$\;
      }{
      \Return{$\bot$}\;
      }
    }
   }
   \Return{$\rf$}\;
   }
  \caption{step1()}
  \label{lwwreg_uniq:1}
\end{procedure}

\begin{procedure}
  {\footnotesize
   \SetKwInOut{KwInput}{Input}
   \SetKwInOut{KwOutput}{Output}
   \KwIn{A MVReg CRDT history $\hist = \tup{\Op, \ro}$}
   \KwOut{$\rf$}
   \BlankLine
   $\rf \leftarrow \emptyset$\;
   \ForEach{$\xvar \in \vars{\hist}$}{
    \ForEach{$\op_2 \in \Op$ s.t. $\op_2 = \tup{Read(x), A}$}{
    \ForEach{$a \in A$}{
    
    \eIf{$\exists \op_1 \in \Op$ s.t. $\op_1 = \tup{Write(x, a), \bot}$}{
      $\rf \leftarrow \rf \cup \{\tup{\op_1, \op_2}\}$\;
      }{
      \Return{$\bot$}\;
      }
    }
    }
   }
   \Return{$\rf$}\;
   }
  \caption{step1()}
  \label{mvreg_uniq:1}
\end{procedure}

\begin{procedure}
  {\footnotesize
   \SetKwInOut{KwInput}{Input}
   \SetKwInOut{KwOutput}{Output}
   \KwIn{A RGA CRDT history $\hist = \tup{\Op, \ro}$}
   \KwOut{$\rf$}
   \BlankLine
   $\rf \leftarrow \emptyset$\;
   \ForEach{$\xvar \in \vars{\hist}$}{
    \ForEach{$\op_2 \in \Op$ s.t $\op_2 = \tup{Read(), [\ldots,a,\ldots]}/\tup{AddAfter(a, \_), \bot}/\tup{Remove(a), \bot}$}{
      \eIf{$\exists \op_1 \in \Op$ s.t. $\op_1 = \tup{AddAfter(\_, a), \bot}$}{
      $\rf \leftarrow \rf \cup \{\tup{\op_1, \op_2}\}$\;
      }{
      \Return{$\bot$}\;
      }
    }
   }
   \While{true}{
   $\rf_1 \leftarrow \emptyset$\;
   \ForEach{$\xvar \in \vars{\hist}$}{
    \ForEach{$\op_1, \op_2 \in \Op$ s.t. $\tup{\op_1, \op_2} \in (\rf\cup\ro)^+$ and $\op_1 = \tup{AddAfter(\_, a), \bot}$ and $\op_2 = \tup{Read(), A}$ and $a \not\in A$}{
    \eIf{$\exists \op_3 \in \Op$ s.t. $\op_3 = Remove(a)$}{
    
      $\rf_1 \leftarrow \rf_1 \cup \{\tup{\op_1, \op_3}\}$\;
      }{
      \Return{$\bot$}\;
      }
    }
   }
   \eIf{$\rf_1 = \emptyset$}{
    break\;
   }{
    $\rf \leftarrow \rf \cup \rf_1$\;
   }
   }
   \Return{$\rf$}\;
   }
  \caption{step1()}
  \label{rga:1}
\end{procedure}

\begin{procedure}
  {\footnotesize
   \SetKwInOut{KwInput}{Input}
   \SetKwInOut{KwOutput}{Output}
   \KwIn{$\tup{\tup{\Op, \ro}, \rf, \hb}$}
   \KwOut{$\lin$}
   \BlankLine
   $\lin \leftarrow \emptyset$\;
   \ForEach{$\xvar \in \vars{\hist}$}{
    \ForEach{$\tup{\op_1, \op_2} \in \rf$ st. $\op_1 = \tup{Write(x, a), \bot}$}{
     \If{$\exists \op_3 \in \Op$ s.t. $\op_3 = \tup{Write(x, b), \bot}$ and $a \neq b$ and $\tup{\op_3,\tr_2}\in \hb$} {
      $\lin \leftarrow \lin \cup \{\tup{\op_3, \op_1}\}$\;
     }
    }
   }
   \Return{$\lin$}\;
   }
  \caption{step2()}
  \label{lwwreg_uniq:2}
\end{procedure}

\begin{procedure}
  {\footnotesize
   \SetKwInOut{KwInput}{Input}
   \SetKwInOut{KwOutput}{Output}
   \KwIn{$\tup{\tup{\Op, \ro}, \rf, \hb}$}
   \KwOut{$\lin$}
   \BlankLine
   $\lin \leftarrow \emptyset$\;
   \ForEach{$\xvar \in \vars{\hist}$}{
    \ForEach{$\tup{\op_1, \op_2} \in \rf$ st. $\op_1 = \tup{Write(x, A), \bot}$}{
     \If{$\exists \op_3 \in \Op$ s.t. $\op_3 = \tup{Write(x, b), \bot}$ and $b \not\in A $ and $\tup{\op_3,\tr_2}\in \hb$} { %\Path{\tr_2}{E_1^+}{\tr_3}, \Path{\tr_1}{\wro[\xvar]}{\tr_3}
      $\lin \leftarrow \lin \cup \{\tup{\op_3, \op_1}\}$\;
     }
    }
   }
   \Return{$\lin$}\;
   }
  \caption{step2()}
  \label{mvreg_uniq:2}
\end{procedure}

\begin{procedure}
  {\footnotesize
   \SetKwInOut{KwInput}{Input}
   \SetKwInOut{KwOutput}{Output}
   \KwIn{$\tup{\tup{\Op, \ro}, \rf, \hb}$}
   \KwOut{$\lin$}
   \BlankLine
   $\rf_1 \leftarrow \{\tup{\op_1, \op_2} \in \rf | \op_1 = AddAfter(\_, \_), \op_2 = AddAfter(\_, \_)\}$\;
    \ForEach{$\op_1 \neq \op_2 \in \Op$ st. $\op_1 = \tup{AddAfter(a, \_), \bot}$ and $\op_2 = \tup{AddAfter(a, \_), \bot}$}{
     \If{$\exists \op'_1, \op'_2 \in \Op$ s.t. $\tup{\op_1, \op'_1}, \tup{\op_2, \op'_2} \in \rf^*_1$ s.t. $\op_1 = AddAfter(\_, b), \op_2 = AddAfter(\_, c)$ s.t. $\exists \tup{Read(), [\ldots, b, \ldots, c, \ldots]}$ }{
      $\lin \leftarrow \lin \cup \{\tup{\op_2, \op_1}\}$\;
     }
    }
   \Return{$\lin$}\;
   }
  \caption{step2()}
  \label{mvreg_uniq:2}
\end{procedure}



%!TEX root = ../../Thesis.tex
\subsection{Replicated Counters}
\label{sec:counter}

In this section, we show that checking consistency for the replicated counter datatype becomes polynomial time assuming the number of replicas in the input history is fixed (i.e., the width of the replica order $\RO$ is fixed). We present an algorithm which constructs a valid happens-before order (note that the semantics of the replicated counter doesn't constrain the linearization order) incrementally, following the replica order. At any time, the happens-before order is uniquely determined by a \emph{prefix mapping} that associates to each replica a \emph{prefix} of the history, i.e., a set of operations which is downward-closed w.r.t. replica order (i.e., if it contains an operation it contains all its $\RO$ predecessors). This models the fact that the replica order is included in the happens-before and therefore, if an operation $o_1$ happens-before another operation $o_2$, then all the $\RO$ predecessors of $o_1$ happen-before $o_2$. The happens-before order can be extended in two ways: (1) adding an operation issued on the replica $i$ to the prefix of replica $i$, or (2) ``merging'' the prefix of a replica $j$ to the prefix of a replica $i$ (this models the delivery of an operation issued on replica $j$ and all its happens-before predecessors to the replica $i$). Verifying that an extension of the happens-before is valid, i.e., that the return values of newly-added {\sf read} operations satisfy the \textsc{RetvalCounter} axiom, doesn't depend on the happens-before order between the operations in the prefix associated to some replica (it is enough to count the {\sf inc} and {\sf dec} operations in that prefix). Therefore, the algorithm can be seen as a search in the space of prefix mappings. If the number of replicas in the input history is fixed, then the number of possible prefix mappings is polynomial in the size of the history, which implies that the search can be done in polynomial time.

%\setlength{\textfloatsep}{3pt}
\begin{algorithm}[t]
  {\footnotesize\SetKwInOut{KwInput}{Input}
\SetKwInOut{KwOutput}{Output}
\KwIn{History $\hist = (\Op, \ro)$, prefix map $m$, and set $\mathit{seen}$ of invalid prefix maps}
\KwOut{$\mathit{false}$ if there exists no read-from and happens-before relations $\rf$ and $\hb$ such that $m\subseteq \hb$, and $\tup{\hist,\rf,\hb}$ satisfies the counter axioms.}
\BlankLine
\lIf{$m$ is complete}{\Return{true}}
 \ForEach{replica $i$}{
   \ForEach{replica $j \neq i$}{
     $m'\leftarrow m[i \leftarrow m(i) \cup m(j)]$\;
     \If{$m'\not\in seen$ and \emph{$\mathsf{checkCounter}(\hist,m',\mathit{seen})$}}{
       \Return{true}\;
     }
     $\mathit{seen}\leftarrow\mathit{seen}\cup\{m'\}$\;
   }
   \If{$\exists o_1.\ \ro^1({\sf last}_i(m),o_1)$}{ %$\exists \op_1 \in (\Op|_{R_i} \setminus M[R_i])$ such that $\tup{\op_2, \op_1} \in \ro \Rightarrow \op_2 \in M[R_i]$
     \If{$\mathsf{meth}(\op_1) = \mathsf{read}$ and $\mathsf{arg}(\op_1) = \xvar \land \mathsf{ret}(\op_1) \neq |\{\op \in m[i] | \op = {\sf inc}(x)\}| - |\{\op \in m[i] | \op = {\sf dec}(x)\}|$}{
         \Return{false}\;
       }
       $m'\leftarrow m[i \leftarrow m(i) \cup \{o_1\}]$\;
       \If{$m' \not\in seen$ and $\mathsf{checkCounter}(\hist, m', \mathit{seen})$}{
         \Return{true}\;
       }
       $\mathit{seen}\leftarrow\mathit{seen}\cup\{m'\}$\;
   }
 }
 \Return{false}\;
}
  \caption{The procedure $\mathsf{checkCounter}$, where $\ro^1$ denotes immediate $\ro$-successor, and $f[a\leftarrow b]$ updates function $f$ with mapping $a \mapsto b$.}
  \label{countercrdtalgo:main}
\end{algorithm}


Let $\hist = (\Op, \ro)$ be a history.
To simplify the notations, we assume that the replica order is a union of sequences, each sequence representing the operations issued on the same replica.
Therefore, each operation $o\in \Op$ is associated with a replica identifier ${\sf rep}(o)\in [1..n_\hist]$, where $n_\hist$ is the number of replicas in $\hist$.

A \emph{prefix} of $\hist$ is a set of operation $\Op'\subseteq \Op$ such that all the $\ro$ predecessors of operations in $\Op'$ are also in $\Op'$, i.e., $\forall \op\in \Op.\ \ro^{-1}(\op)\in \Op$. Note that the union of two prefixes of $\hist$ is also a prefix of $\hist$. The \emph{last operation} of replica $i$ in a prefix $\Op'$ is the $\ro$-maximal operation $o$ with ${\sf rep}(o)=i$ included in $\Op'$.
A prefix $\Op'$ is called \emph{valid} if $(\Op',\ro')$, where $\ro'$ is the projection of $\ro$ on $\Op'$, is admitted by the replicated counter.

A \emph{prefix map} is a mapping $m$ which associates a prefix of $\hist$ to each replica $i\in [1..n_\hist]$.
%The \emph{last operation} of a replica $i$ in a prefix map $m$ is the maximal $\ro$ operation o
Intuitively, a prefix map defines for each replica $i$ the set of operations which are ``known'' to $i$, i.e., happen-before the last operation of $i$ in its prefix. Formally, a prefix map $m$ is \emph{included} in a happens-before relation $\hb$, denoted by $m\subseteq \hb$, if for each replica $i\in [1..n_\hist]$, $\hb(o,o_i)$ for each operation in $o\in m(i)\setminus\{o_i\}$, where $o_i$ is the last operation of $i$ in $m(i)$. We call $o_i$ the \emph{last operation} of $i$ in $m$, and denoted it by ${\sf last}_i(m)$.
A prefix map $m$ is \emph{valid} if it associates a valid prefix to each replica, and \emph{complete} if it associates the whole history $\hist$ to each replica $i$.

Algorithm~\ref{countercrdtalgo:main} lists our algorithm for checking consistency of replicated counter histories. It is defined as a recursive procedure $\mathsf{checkCounter}$ that searches for a sequence of valid extensions of a given prefix map (initially, this prefix map is empty) until it becomes complete. The axiom \textsc{RetvalCounter} is enforced whenever extending the prefix map with a new {\sf read} operation (when the last operation of a replica $i$ is ``advanced'' to a {\sf read} operation). The following theorem states of the correctness of the algorithm.

\vspace{-2mm}
\begin{theorem}

  $\mathsf{checkCounter}(\hist,\emptyset,\emptyset)$ returns $\mathit{true}$ if{f} the input history is consistent.

\vspace{-2mm}
\end{theorem}

When the number of replicas is fixed, the number of prefix maps becomes polynomial in the size of the history. This follows from the fact that prefixes are uniquely defined by their $\ro$-maximal operations, whose number is fixed.

\vspace{-2mm}
\begin{corollary}

  The admissibility problem for replicated counters is polynomial-time when the number of replicas is fixed.

\vspace{-2mm}
\end{corollary}


% For instance, given the history in Figure~\ref{ser_algo_example:1}, the set of transactions $\{\op_0,\op_1,\op_2\}$ is a prefix with boundary $\{\op_1,\op_2\}$ (the latter is an antichain of the session order).


%!TEX root = Thesis.tex

\section{Polynomial-Time Algorithms for Sets and Flags}
\label{sec:ptime:sets}

While Theorem~\ref{thm:3sat-to-flags} shows that the admissibility problem is NP-complete for replicated sets and flags even if the number of replicas is fixed, we show that this problem becomes polynomial time when additionally, the number of values added to the set, or the number of flags, is also fixed. Note that this doesn't limit the number of operations in the input history which can still be arbitrarily large. In the following, we focus on the Add-Wins Set, the other cases being very similar.

We propose an algorithm for checking consistency which is actually an extension of the one presented in Section~\ref{sec:counter} for replicated counters.
%In this section, we show that checking consistency for the replicated OR-Set datatype becomes polynomial time assuming the number of replicas and number of values added in the set in the input history is fixed (i.e. the width of the replica order $\RO$ is fixed and total number of values added in the set is fixed). We present a similar algorithm to replicated Counter where we constructs a happens-before order incrementally by progressing a replica or merging a replica to another.
The additional complexity in checking consistency for the Add-Wins Set comes from the validity of {\sf contains}($x$) return values which requires identifying the maximal predecessors in the happens-before relation that add or remove $x$ (which are not necessarily the maximal $\hb$-predecessors all-together). In the case of counters, it was enough just to count happens-before predecessors.
%We use the similar notations from Counter CRDT. But the extra work here is, while verifying that an extension of the happens-before is valid, i.e., the return values of newly-added {\sf read} operations satisfy the {\sc RetvalSet} axiom, it depends on the maximal {\sf add} and {\sf remove} operations in that prefix.
Therefore, we extend the algorithm for replicated counters such that along with the prefix map, we also keep track of the $\hb$-maximal {\sf add}($x$) and {\sf remove}($x$) operations for each element $x$ and each replica $i$.
When extending a prefix map with a {\sf contains} operation, these $\hb$-maximal operations (which define a witness for the read-from relation) are enough to verify the {\sc RetValSet} axiom. Extending the prefix of a replica with an {\sf add} or {\sf remove} operation (issued on the same replica), or by merging the prefix of another replica, may require an update of these $\hb$-maximal predecessors.
% Appendix~\ref{sec:ptime:sets:appendix} describes the details of this algorithm.

When the number of replicas and elements are fixed, the number of read-from maps is polynomial in the size of the history — recall that the number of operations associated by a read-from map to a replica and set element is bounded by the number of replicas. Combined with the number of prefix maps being polynomial when the number of replicas is fixed, we obtain the following result.

\vspace{-.5mm}
\begin{theorem}
  \label{thm:ptime:sets}

  Checking whether a history is admitted by the Add-Wins Set, Remove-Wins Set, Enable-Wins Flag, or the Disable-Wins Flag is polynomial time provided that the number of replicas and elements/flags is fixed.

\vspace{-1.5mm}
\end{theorem}


We propose a general algorithm to verify a CRDT history.

\begin{definition}
 A visible prefix of a replica over multiple replicas is the union of a prefix of each replica with the respective valid $\hb, \rf$(also, $\lin$ if required) restricted to that union.
\end{definition}

\begin{algorithm}[ht]
 \SetKwInOut{KwInput}{Input}
 \SetKwInOut{KwOutput}{Output}
 \KwIn{A CRDT history $\hist = \tup{\Op, \RO}$}
 \KwOut{$\mathit{true}$ iff $\hist$ valid history}
 \BlankLine
 $PrefixMap \leftarrow$ for each replica initiate with empty prefix\;
 \While{there is an unfinished prefix in replica map}{
  $R_1 \leftarrow$ replica with unfinished replica map\;
  \eIf{nondeterministically}{
   // receive update from other replica\;
   // i.e. the other replica prefix happens-before the next operations in $R_1$\;
   $R_2 \leftarrow$ nondeterministically choose an replica which has new operations to share\;
   take the visibility map of $R_2$ and \emph{merge} with $R_1$\;
  }{
   // progress current replica\;
   // include the next operation in the current replica\;
   \If{next operation is a Read}{
    check the read values are consistent to current prefix map of $R_1$\;
    // linearization may need to be updated\;
   }
   \If{next operation is an Update}{
    update the prefix map accordingly\;
   }
   // note, a operation can both be Read and Update\;
  }
 }
 \caption{Verifying CRDT history}
 \label{genalgo:1}
\end{algorithm}

\begin{proof}
 To prove the soundness of the algorithm we need to show the algorithm always extends a valid prefix map to another valid prefix map.

 Note, a prefix map is invalid if it is an inconsistent Read.

 While merging two prefixes, the merged prefix does not add a new Read operation. So it is always okay to do this.

 While advancing a replica, only Update operation does not really make a valid. The algorithm only adds a Read operation, while advancing a replica with a Read operation. The algorithm checks for inconsistencies before adding such Read operation.

 So at every step, the algorithm advances with a valid prefix map. If the algorithm is able to reach a successful run, that means, there is a valid CRDT execution of the history.

 To show the completeness of the algorithm, we have to show for every valid history the algorithm can have a successful run. If there is a successful run, that means, there is a valid happens-before $\hb$ relation and it induces a partial order over the replica operations. The algorithm can follow the topological extension of that $\hb$ and have a successful run.
\end{proof}


\section{Verification complexities}

\begin{itemize}
 \item Unique value writes or given $\rf$ for last-writer-wins and multi-value register are in polynomial time. Because it induces unique $\rf$ relation.

 \item Counter with a bounded number of replicas is in P.

    In the general algorithm, it is easy to see, for counter, we don't need to store the actual $\hb$ relation of a prefix. Now there are only polynomial many prefixes of a history of a bounded number of replicas. So to determinize the algorithm, it will require to explore at most each replica with all possible prefix. Since there is a bounded number of replicas, the total configuration of each replica with prefix becomes polynomial size. So the algorithm quits after polynomial many steps.

 \item Replicated growing array. We can figure out the $\rf$ deterministically because of unique add, remove and read in RGA. Then we can deterministically find out the $\hb$ and $\lin$ relations. If we found a valid $\lin$ then, the RGA history is valid.
\end{itemize}

\begin{algorithm}[ht]
 {\footnotesize
  \SetKwInOut{KwInput}{Input}
  \SetKwInOut{KwOutput}{Output}
  \KwIn{A RGA history $\hist = \tup{\Op, \RO}$}
  \KwOut{$\mathit{true}$ iff $\hist$ valid history}
  \BlankLine
  \ForEach{$\op_1, \op_2$} {
   \If{$\op_1$: AddAfter(\_, a) $\land$ $\op_2$: AddAfter(a, \_)/Remove(a)/Read contains a} {
    $\rf \rightarrow \rf \cup \{\tup{\op_1, \op_2}\}$\;
   }
  }
  $\hb \rightarrow \rf$\;
  $\lin \rightarrow \hb$\;
  \While{true} {
   \If{$\op_1$: AddAfter(a, b), $\op_2$: AddAfter(a, c), $\lin$ does not relate $\op_1, \op_2$, $\exists$ Read where b happens after c}{
    $\lin \rightarrow \lin \cup \{\tup{\op_1, \op_2}\}$
   }
  }
  \eIf{$\lin$ is cyclic} {
   \Return{false}\;
  }{
   \Return{true}\;
  }}
 \caption{Verifying RGA history}
 \label{ccalgo:1}
\end{algorithm}
