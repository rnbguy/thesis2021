%!TEX root = ../../Thesis.tex

\subsection{Sets and Flags}
\label{sec:ptime:sets}

While Theorem~\ref{thm:3sat-to-flags} shows that the admissibility problem is NP-complete for replicated sets and flags even if the number of replicas is fixed, we propose a sound algorithm like replicated counter which runs in polynomial time when additionally, the number of values added to the set, or the number of flags, is also fixed. Note that this doesn't limit the number of operations in the input history which can still be arbitrarily large. In the following, we focus on the Add-Wins Set, the other cases being very similar.

The algorithm for checking consistency which is actually an extension of the one presented in Section~\ref{sec:counter} for replicated counters.
%In this section, we show that checking consistency for the replicated OR-Set datatype becomes polynomial time assuming the number of replicas and number of values added in the set in the input history is fixed (i.e. the width of the replica order $\RO$ is fixed and total number of values added in the set is fixed). We present a similar algorithm to replicated Counter where we constructs a happens-before order incrementally by progressing a replica or merging a replica to another.
The additional complexity in checking consistency for the Add-Wins Set comes from the validity of {\sf contains}($x$) return values which requires identifying the maximal predecessors in the happens-before relation that add or remove $x$ (which are not necessarily the maximal $\hb$-predecessors all-together). In the case of counters, it was enough just to count happens-before predecessors.
%We use the similar notations from Counter CRDT. But the extra work here is, while verifying that an extension of the happens-before is valid, i.e., the return values of newly-added {\sf read} operations satisfy the {\sc RetvalSet} axiom, it depends on the maximal {\sf add} and {\sf remove} operations in that prefix.
Therefore, we extend the algorithm for replicated counters such that along with the prefix map, we also keep track of the $\hb$-maximal {\sf add}($x$) and {\sf remove}($x$) operations for each element $x$ and each replica $i$.
When extending a prefix map with a {\sf contains} operation, these $\hb$-maximal operations (which define a witness for the read-from relation) are enough to verify the {\sc RetValSet} axiom. Extending the prefix of a replica with an {\sf add} or {\sf remove} operation (issued on the same replica), or by merging the prefix of another replica, may require an update of these $\hb$-maximal predecessors.
% Appendix~\ref{sec:ptime:sets:appendix} describes the details of this algorithm.


To represent the maximal $\hb$-predecessors, we use a mapping $u$, called \emph{read-from map}, that associates a set of operations {\sf add}($x$) and {\sf remove}($x$) on different replicas to each replica $i$ and element $x$. Note that two operations on the same replica are necessarily related by $\hb$ and cannot be both maximal. A pair of prefix-map $m$ and read-from map $u$ defines a partial read-from relation that associates all the operations in $u(x,i)$ to the last operation of $i$, i.e., ${\sf last}_i(m)$, if this is a {\sf contains}($x$) operation. For a given read-from relation $\rf$, $\tup{m,u}\subseteq \rf$ denotes the fact that this partial read-from relation is included in $\rf$. A prefix $m(i)$ is called valid in the context of a read-from map $u$ if it is admitted by the Add-Wins Set with a read-from relation $\rf$ such that $\tup{m,u}\subseteq \rf$. A pair $\tup{m,u}$ is called valid if $m(i)$ is valid for each replica $i$.

Algorithm~\ref{orsetcrdtalgo:main} lists our algorithm for checking consistency of Add-Wins Set histories. As for replicated counters, it is defined as a recursive procedure $\mathsf{CheckAWSet}$ that searches for a sequence of valid extensions of a given prefix map and read-from map (initially, both of them are empty) until the prefix map becomes complete.
%To progress on {\sf add} or {\sf remove} operation, the algorithm removes the old concurrent updates because they are not latest anymore and the current one as latest. Merging the prefixes are done as Counters. To merge the latest concurrent updates, for each values the algorithm sets it to the maximal updates in both of the replicas.

\begin{algorithm}
 {\footnotesize
  \SetKwInOut{KwInput}{Input}
  \SetKwInOut{KwOutput}{Output}
  \KwIn{A history $\hist = (\Op, \ro)$, a prefix map $m$, a read-from map $u$, and a set $\mathit{seen}$ of invalid prefix map and read-from map pairs.}
  \KwOut{$\mathit{true}$ iff there exists a read-from relation $\rf$ and a happens-before order $\hb$ such that $m\subseteq \hb$, $\tup{m,u}\subseteq \rf$, and $\tup{\hist,\rf,\hb}$ satisfies the replicated Add-Wins Set axioms.}
  \BlankLine
  \If{$m$ is complete}{
   \Return{true}\;
  }
  \ForEach{replica $i$}{
   \ForEach{replica $j \neq i$}{
    $m'\leftarrow m[i \leftarrow m(i) \cup m(j)]$\;
    $u'(x) \leftarrow u(x)[i \leftarrow (u(x, i) \setminus (m(j) \setminus u(x, j)) \cup (u(x,j) \setminus (m(i) \setminus u(x, i)))]$\;
    \If{$\tup{m', u'}\not\in seen$ and \emph{$\mathsf{CheckAWSet}(\hist,m',u',\mathit{seen})$}}{
     \Return{true}\;
    }
    $\mathit{seen}\leftarrow\mathit{seen}\cup\{\tup{m', u'}\}$\;
   }
   \If{$\exists o_1.\ \ro^1({\sf last}_i(m),o_1)$}{ %$\exists \op_1 \in (\Op|_{R_i} \setminus M[R_i])$ such that $\tup{\op_2, \op_1} \in \ro \Rightarrow \op_2 \in M[R_i]$
    $u' \leftarrow u$\;
    \eIf{$\mathsf{meth}(\op_1) = \mathsf{contains}$}{
     \If{$\mathsf{ret}(\op_1) = \mathit{true} \Leftrightarrow \not\exists \op_2 \in u(\mathsf{arg}(\op_1),i)$ st. ${\sf meth}(\op_2) = {\sf add} \land {\sf arg}(\op_2) = \mathsf{arg}(\op_1)$}{
      \Return{false}\;
     }
    }{
     $u'(\mathsf{arg}(\op_1),i) \leftarrow \{\op_1\}$\;
    }
    $m'\leftarrow m[i \leftarrow m(i) \cup \{o_1\}]$\;
    \If{$\tup{m', u'} \not\in seen$ and $\mathsf{CheckAWSet}(\hist, m', u', \mathit{seen})$}{
     \Return{true}\;
    }
    $\mathit{seen}\leftarrow\mathit{seen}\cup\{\tup{m', u'}\}$\;
   }
  }
  \Return{false}\;
 }
 \caption{The procedure $\mathsf{CheckAWSet}$ for checking consistency of Add-Wins Set.
% $\ro^1$ denotes the immediate successor relation of $\ro$.
% Also, for a mapping $f$, $f[a\leftarrow b]$ denotes the function $f'$ with $f'(x)=f(x)$ for all $x\neq a$, and $f'(a)=b$.
}
 \label{orsetcrdtalgo:main}
\end{algorithm}


%Algorithm~\ref{orsetcrdtalgo:main} lists our algorithm for checking consistency of replicated OR-Set histories. It is defined as a recursive procedure $\mathsf{checkORSet}$ that searches for a sequence of valid extensions of a given prefix map and a latest concurrent value update map (initially, both of them are empty) until the prefix map becomes complete. The axiom \textsc{RetvalSet} is enforced whenever extending the prefix map with a new {\sf contain} operation (when the last operation of a replica $i$ is ``advanced'' to a {\sf read} operation).
The following theorem states the correctness of the algorithm.

\begin{theorem}

  $\mathsf{CheckAWSet}(\hist,\emptyset,\emptyset,\emptyset)$ returns $\mathit{false}$ if the input history is not consistent for the Add-Wins Set.

\end{theorem}

When the number of replicas and elements are fixed, the number of read-from maps is polynomial in the size of the history — recall that the number of operations associated by a read-from map to a replica and set element is bounded by the number of replicas. Combined with the number of prefix maps being polynomial when the number of replicas is fixed, we obtain the following result.

% Theorem~\ref{thm:ptime:sets} in Section~\ref{sec:ptime:sets} follows from this result, so long as the number of replicas and set/flag elements are fixed.

We present a history of a replicated flag in figure \ref{correction:fig2} which will not be explored by this algorithm. The argument is very similar as the one for replicated counter (figure \ref{correction:fig1}).

% \vspace{-.5mm}
% \begin{theorem}
%   \label{thm:ptime:sets}

%   Checking whether a history is admitted by the Add-Wins Set, Remove-Wins Set, Enable-Wins Flag, or the Disable-Wins Flag is polynomial-time provided that the number of replicas and elements/flags is fixed.

% \vspace{-1.5mm}
% \end{theorem}
