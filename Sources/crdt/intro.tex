%!TEX root = Thesis.tex

%\section{Introduction}
%\label{sec:crdt:intro}

%Recent distributed systems have introduced variations of familiar abstract data types (ADTs) like counters, registers, flags, and sets, that provide high availability and partition tolerance. These \emph{conflict-free replicated data types} (CRDTs)~\cite{DBLP:conf/sss/ShapiroPBZ11} efficiently resolve the effects of concurrent updates to replicated data. Naturally, they weaken consistency guarantees to achieve availability and partition-tolerance, and various notions of \emph{weak consistency} capture such guarantees~\cite{DBLP:conf/pdis/TerryDPSTW94, DBLP:conf/sosp/TerryTPDSH95, DBLP:conf/popl/MansonPA05, DBLP:journals/ftpl/Burckhardt14, DBLP:conf/popl/BurckhardtGYZ14}.

In this chapter we study the tractability of runtime CRDT consistency checking: deciding whether a given execution of a CRDT is consistent with its specification. 
Our setting captures executions across a set of replicas as per-replica sequences of operations called \emph{histories}. Roughly speaking, a history is \emph{consistent} so long as each operation’s return value can be justified according to the operations that its replica has observed so far. In the setting of CRDTs, the determination of a replica’s observations is essentially an implementation choice: replicas are only obliged to observe their own operations, and the predecessors of those it has already observed. This relatively-weak constraint on replicas’ observations makes the CRDT consistency checking problem unique.

%\begin{figure}[t]
%  \centering
%  \setlength{\tabcolsep}{1em}
\renewcommand{\arraystretch}{1.2}
\begin{tabular}{lr}
  \toprule
  Data Types & Complexity \\
  \cmidrule(lr){1-2}
  Add-Wins Set, Remove-Wins Set         & {\sc np}-complete \\
  Enable-Wins Flag, Disable-Wins Flag   & {\sc np}-complete \\
  Sets \& Flags — with bounded domains  & {\sc ptime} \\
  Last-Writer-Wins Register ({\sc lww}) & {\sc np}-complete \\
  Multi-Value Register ({\sc mvr})      & {\sc np}-complete \\
  Registers – with unique values        & {\sc ptime} \\
  Replicated Counters                   & {\sc np}-complete \\
  Counters – with bounded replicas      & {\sc ptime} \\
  Replicated Growable Array ({\sc rga}) & {\sc ptime} \\
  \bottomrule
\end{tabular}

%  \caption{The complexity of consistency checking for various replicated data types. We demonstrate intractability and tractability results in Sections~\ref{sec:intractability} and~\ref{sec:algorithms}, respectively.}
%  \label{fig:results}
%\end{figure}

%To precisely characterize the consistency of various CRDTs, and facilitate symbolic reasoning, we develop novel logical characterizations to capture their guarantees. Our logical models
We present logical characterizations of CRDTs, which are built on a notion of \emph{abstract execution}, which relates the operations of a given history with three separate relations: a \emph{read-from} relation, governing the observations from which a given operation constitutes its own return value; a \emph{happens-before} relation, capturing the causal relationships among operations; and a \emph{linearization} relation, capturing any necessary arbitration among non-commutative effects which are executed concurrently, e.g.,~following a \emph{last-writer-wins} policy. Accordingly, we capture data type specifications with logical axioms interpreted over the read-from, happens-before, and linearization relations of abstract executions, reducing the consistency problem to: does there exist an abstract execution over the given history which satisfies the axioms of the given data type?

We demonstrate the intractability of several replicated data types by reduction from propositional satisfiability (SAT) problems. In particular, we consider the 1-in-3 SAT problem~\cite{DBLP:books/fm/GareyJ79}, which asks for a truth assignment to the variables of a given set of clauses such that exactly one literal per clause is assigned true. Our reductions essentially simulate the existential choice of a truth assignment with the existential choice of the read-from and happens-before relations of an abstract execution. For a given 1-in-3 SAT instance, we construct a history of replicas obeying carefully-tailored synchronization protocols, which is consistent exactly when the corresponding SAT instance is positive.

Finally, we develop tractable consistency-checking algorithms for individual data types and special cases: replicated growing arrays; multi-value and last-writer-wins registers, when each value is written only once; counters, when replicas are bounded; and sets and flags, when their sizes are also bounded. While the algorithms for each case are tailored to the algebraic properties of the data types they handle, they essentially all function by constructing abstract executions incrementally, processing replicas’ operations in prefix order.

The remainder of this chapter is organized as follows:
% \vspace{-3.5mm}
\begin{itemize}

  \item Section~\ref{sec:consistency} presents logical characterizations of consistency for the replicated register, flag, set, counter, and array data types;

  \item Section~\ref{sec:intractability} introduces reductions from propositional satisfiability problems to consistency checking to demonstrate intractability for replicated flags, sets, counters, and registers; and

  \item Section~\ref{sec:algorithms} defines polynomial time consistency-checking algorithms for replicated growable arrays, registers, when written values are unique, counters, when replicas are bounded, and sets and flags, when their sizes are also bounded.

\end{itemize}
% \vspace{-1.5mm}
Section~\ref{sec:crdt:related} overviews related work, and Section~\ref{sec:crdt:conclusion} concludes.
