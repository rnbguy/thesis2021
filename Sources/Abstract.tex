%!TEX root = Thesis.tex

\begin{center}
  \textsc{Abstract}
\end{center}
%
\noindent
%



% crdt
% Recent distributed systems have introduced variations of familiar abstract data types 
% (ADTs) like counters, registers, flags, and sets, that provide high availability and 
% partition tolerance. These \emph{conflict-free replicated data types} (CRDTs) utilize 
% mechanisms to resolve the effects of concurrent updates to replicated data. Naturally 
% these objects weaken their consistency guarantees to achieve availability and 
% partition-tolerance, and various notions of \emph{weak consistency} capture 
% those guarantees.

% In this work we study the tractability of CRDT-consistency checking. 
% To capture guarantees precisely, and facilitate symbolic reasoning, we propose 
% novel logical characterizations. By developing novel reductions from propositional 
% satisfiability problems, and novel consistency-checking algorithms, we discover both 
% positive and negative results. In particular, we show intractability for replicated flags, 
% sets, counters, and registers, yet tractability for replicated growable arrays. 
% Furthermore, we demonstrate that tractability can be redeemed for registers when each 
% value is written at most once, for counters when the number of replicas is fixed, and 
% for sets and flags when the number of replicas and variables is fixed.

% txnkv
% Transactions simplify concurrent programming by enabling computations on shared 
% data that are isolated from other concurrent computations and resilient to failures. 
% Modern databases provide different consistency models for transactions corresponding 
% to different tradeoffs between consistency and availability. In this work, we investigate 
% the problem of checking whether a given execution of a transactional database adheres to 
% some consistency model. We show that consistency models like read committed, read atomic, 
% and causal consistency are polynomial time checkable while prefix consistency and snapshot 
% isolation are NP-complete in general. These results complement a previous NP-completeness 
% result concerning serializability. Moreover, we identify a generic class of executions, 
% for which checking consistency models which are NP-complete in general becomes polynomial 
% time. We evaluate the scalability of our algorithms in the context of several production 
% databases.

% monkeydb
% Modern applications, such as social networking systems and e-commerce platforms
% are centered around using large-scale storage systems for storing and retrieving
% data. In the presence of concurrent accesses, these storage systems trade off isolation
% for performance. The weaker the isolation level, the more behaviors a storage
% system is allowed to exhibit and it is up to the developer to ensure that their
% application can tolerate those behaviors. However, these weak behaviors only
% occur rarely in practice, that too outside the control of the application, 
% making it difficult for developers to test the robustness of their 
% code against weak isolation levels. 

% This paper presents MonkeyDB, a mock storage system for testing storage-backed
% applications. MonkeyDB supports a Key-Value interface as well as SQL queries
% under multiple isolation levels. It uses a logical specification of the isolation
% level to compute, on a read operation, the set of all possible return values.
% MonkeyDB then returns a value randomly from this set. 
% We show that MonkeyDB provides 
% good coverage of weak behaviors, which is complete in the limit. We test a
% variety of applications for assertions that fail only under weak isolation.
% MonkeyDB is able to break each of those assertions in a small number of attempts. 

Recent distributed systems have introduced various abstract concepts like 
\emph{conflict-free replicated data types} (CRDTs), transaction, weak consistency and
isolation level.

CRDTs like counters, registers, flags, and sets provide high availability
and partition tolerance. They utilize nontrivial mechanisms to resolve the effects of
concurrent updates to replicated data. Naturally these objects weaken their 
consistency guarantees to achieve availability and partition-tolerance, and 
various notions of \emph{weak consistency} capture those guarantees.

Transactions simplify concurrent programming by enabling computations on shared data
that are isolated from other concurrent computations and resilient to failures.
Modern databases provide different consistency models or isolation level for transactions 
corresponding to different tradeoffs between consistency and availability.

This thesis studies the tractability of CRDT-consistency checking and transactional 
database checking for an insolation level. In both cases, to capture guarantees precisely,
and facilitate symbolic reasoning, we propose novel logical characterizations.
By developing novel reductions from propositional satisfiability problems, and novel
consistency-checking algorithms, we discover both positive and negative results. We also 
demonstrate that tractability for hard cases can be redeemed if some parameter is bounded,
usually the number of session or replicas in the network.

Lastly, this thesis presents MonkeyDB. a mock storage system for testing storage-backed
applications. MonkeyDB supports a Key-Value interface as well as SQL queries
under multiple isolation levels. It uses a logical specification of the isolation
level to compute, on a read operation, the set of all possible return values.
MonkeyDB then returns a value randomly from this set. 
We show that MonkeyDB provides 
good coverage of weak behaviors, which is complete in the limit. We test a
variety of applications for assertions that fail only under weak isolation.
MonkeyDB is able to break each of those assertions in a small number of attempts. 