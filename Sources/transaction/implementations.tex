%!TEX root = ../../Thesis.tex


Even though we describe the Serializability checking algorithm theoretically in subsection \ref{ssec:ser_checking} and showed the reductions for Prefix Consistency in subsection \label{ssec:pc} and Snapshot Isolation in subsection \label{ssec:si}, we should discuss about our actual implementations before discussing our experimental results.

After processing the history, we maintain $\so$, $\wro_{x}$ as relations in hashmaps for faster access. For relation set related operations, we implemented optimized algorithms for this hashmaps such as computing transitivity on the fly, checking inclusion, finding cycles. 

\subsection{Implementation for Serializability}\label{ssec:imp-ser}
We follow our algorithm \ref{seralgo:2} to implement this. The main difference to the pseudo-code is to reduce the storage complexity while doing the DFS search (demonstrated in figure \ref{ser_algo_example:3}), we maintain a state of the search in an optimized data structure. This data structure includes the sequence of choices till that point and the active variables $x$ which has a active $\wro_{x}$ between a \textsf{Write} inside of current serializable prefix and a \textsf{Read} outside of the prefix.

TODO: example

When we add a new transaction in the serializable prefix, we add it to the sequence of choices and update the active variable set.

When we backtrack, we read our history and remove the last choice of transaction and read the history and restore the previous active variable set.


\subsection{Implementation for Prefix Consistency}\label{ssec:imp-pc}

Even though we have a reduction for Prefix Consistency to Serializability, we implement a direct algorithm for it. We perform the reduction on the fly. We implement our serializable algorithm on a \emph{virtual} history where the a splited transaction $T$ is denoted by $(T, Read)$ and $(T, Write)$. 

\subsection{Implementation for Snapshot Isolation}\label{ssec:imp-si}

The algorithm for Snapshot Isolation is implemented similarly as Prefix Consistency. A splited transaction $T$ is denoted by $(T, Read)$ and $(T, Write)$. But two transactions those write on same variable, their split transactions can not interleave. So $\vartriangleright$ also checks if a $(T, Read)$ is includes, there is no $T'$ in the serializable prefix such that $T$ and $T'$ do not write on same variable.
