%!TEX root = Thesis.tex

\section{Consistency Criteria}\label{sec:def}

\subsection{Histories}


% \begin{table*}
%  \centering
%  \resizebox{\textwidth}{!}{
%   \begin{tabular}{ |c c|c c|c c|c c|}
%    \hline
%    \multicolumn{7}{|c}{
%     \shortstack{
%      $\forall (O, \textsf{po}) \in \mathcal{T}, \forall o \in O, o = read(x, n) \wedge \left\{o' \in \textsf{po}^{-1}(o) \mid o' = \_(x, \_) \right\} \neq \emptyset \Rightarrow$ \\
%      $max_{\textsf{po}} \left(\left\{o' \in \textsf{po}^{-1}(o) \mid o' = \_(x, \_) \right\}\right) = \_(x, n)$
%     }
%    }                    & {\textsc{(Int)}}                                                                                                                                        \\
%    \hline
%    \multicolumn{7}{|c}{
%     \shortstack{
%      $\forall T = (O, \textsf{po}) \in \mathcal{T}, \forall x, T \models \texttt{read}(x, n) \Rightarrow max_{\CO} \left(\VIS^{-1}(T) \cap \textsf{Write}_x\right) \models \texttt{write}(x, n)$
%     }
%    }                    & {\textsc{(Ext)}}                                                                                                                                        \\
%    \hline
%    $\SO \subseteq \VIS$ & \textsc{(Session)}      & \VIS is transitive & \textsc{(TransVis)} & $\CO;\VIS \subseteq \VIS$ & \textsc{(Prefix)} & $\VIS = \CO$ & \textsc{(TotalVis)} \\
%    \hline
%    \multicolumn{7}{|c}{
%     $\forall T, S \in \mathcal{T}, \forall x, (T, S \in \textsf{Write}_x \wedge T \neq S) \Rightarrow (T \xrightarrow{\VIS} S \vee S \xrightarrow{\VIS} T)$
%    }                    & {\textsc{(NoConflict)}}                                                                                                                                 \\
%    \hline
%   \end{tabular}
%  }
%  \caption{Consistency axioms, declaring behaviors on schedule}
%  \label{weakconsistency:1}
% \end{table*}



We consider a transactional database storing a set of variables $\Var=\{\xvar,\yvar,\ldots\}$. Clients interact with the database by issuing transactions formed of $\textsf{read}$ and $\textsf{write}$ operations. Assuming an unspecified set of values $\Val$ and a set of operation identifiers $\OId$, we let 
\begin{align*}
 \Op=\set{\rd[\id]{\xvar}{\val},\wrt[\id]{\xvar}{\val}: \id\in\OId, \xvar\in\Var, \val\in \Val}
\end{align*} 
be the set of operations reading a value $\val$ or writing a value $\val$ to a variable $\xvar$. We omit operation identifiers when they are not important.

\begin{definition}
 A \emph{transaction} $\tup{O, \po}$ is a finite set of operations $O$ along with a strict total order $\po$ on $O$, called \emph{program order}.
\end{definition}

We use $\tr$, $\tr_1$, $\tr_2$, $\ldots$ to range over transactions. The set of read, resp., write, operations in a transaction $\tr$ is denoted by $\readOp{\tr}$, resp., $\writeOp{\tr}$. The extension to sets of transactions is defined as usual. Also, we say that a transaction $\tr$ \emph{writes} a variable $\xvar$, denoted by $\writeVar{\tr}{\xvar}$, when $\wrt[\id]{\xvar}{\val}\in \writeOp{\tr}$ for some $\id$ and $\val$. Similarly, a transaction $\tr$ \emph{reads} a variable $\xvar$ when $\rd[\id]{\xvar}{\val}\in \readOp{\tr}$ for some $\id$ and $\val$.

\begin{figure}[t]
 \centering
 \begin{subfigure}{.21\textwidth}
   \centering
  \resizebox{!}{1.3cm}{
   \begin{tikzpicture}[->,>=stealth',shorten >=1pt,auto,node distance=2cm,
     semithick, transform shape]
    \node[draw, rounded corners=2mm] (t1) {\begin{tabular}{l} \texttt{x = 1;} \\ ... \\ \texttt{x = 2;} \end{tabular}};
    \node[draw, rounded corners=2mm, right of = t1] (t2) {\begin{tabular}{l} ... \\ \texttt{read(x);} \end{tabular}};
   \end{tikzpicture}  
  }
%  \caption{Only the lastest write is visible to other transaction}
\caption{}
  \label{rc_eg:1}
 \end{subfigure}
 \hspace{10mm}
 \begin{subfigure}{.1\textwidth}
   \centering
  \resizebox{!}{2cm}{
   \begin{tikzpicture}[->,>=stealth',shorten >=1pt,auto,node distance=2cm,
     semithick, transform shape]
    \node[draw, rounded corners=2mm] (t1) {\begin{tabular}{l} \texttt{x = 1;} \\ ... \\ \texttt{x = 2;} \\ ... \\ \texttt{read(x);}\end{tabular}};
   \end{tikzpicture}  
  }
%  \caption{Always reads the latest write inside a transaction}
\caption{}
  \label{rr_eg:1}
 \end{subfigure}
 \hspace{10mm}
 \begin{subfigure}{.14\textwidth}
   \centering
  \resizebox{!}{1.3cm}{
   \begin{tikzpicture}[->,>=stealth',shorten >=1pt,auto,node distance=2cm,
     semithick, transform shape]
    \node[draw, rounded corners=2mm] (t1) {\begin{tabular}{l} \texttt{x = 1;} \\ ... \\ \texttt{ABORT;} \end{tabular}};
    \node[draw, rounded corners=2mm, right of = t1] (t2) {\begin{tabular}{l} ... \\ \texttt{read(x);} \end{tabular}};
   \end{tikzpicture}  
  }
%  \caption{Aborted transactions are not visible}
\caption{}
  \label{abort:1}
 \end{subfigure}
 \caption{Examples of transactions used to justify our simplifying assumptions (each box represents a different transaction): (a) only the last written value is observable in other transactions, (b) reads following writes to the same variable return the last written value in the same transaction, and (c) values written in aborted transactions are not observable.}
 \label{read_latest}
 \vspace{-3mm}
\end{figure}

To simplify the exposition, we assume that each transaction $\tr$ contains at most one write operation to each variable\footnote{That is, for every transaction $\tr$, and every $\wrt{\xvar}{\val}, \wrt{\yvar}{\val'}\in \writeOp{\tr}$, we have that $\xvar\neq\yvar$.}, and that a read of a variable $\xvar$ cannot be preceded by a write to $\xvar$ in the same transaction\footnote{That is, for every transaction $\tr=\tup{O, \po}$, if $\wrt{\xvar}{\val}\in \writeOp{\tr}$ and there exists $\rd{\xvar}{\val}\in \readOp{\tr}$, then we have that $\tup{\rd{\xvar}{\val},\wrt{\xvar}{\val}}\in \po$}. If a transaction would contain multiple writes to the same variable, then only the last one should be visible to other transactions (w.r.t. any consistency criterion considered in practice). For instance, the \texttt{read(x)} in Figure~\ref{rc_eg:1} should not return 1 because this is not the last value written to {\tt x} by the other transaction. It can return the initial value or 2.
%In figure (\ref{rc_eg:1}), however the two transactions are executed, the operation \texttt{print(x)} in below transaction should not print \texttt{0}. Because \texttt{x=0} is not the latest write in the above transaction.
Also, if a read would be preceded by a write to the same variable in the same transaction, then it should return a value written in the same transaction (i.e., the value written by the latest write to $\xvar$ in that transaction). 
For instance, the \texttt{read(x)} in Figure~\ref{rr_eg:1} can only return 2 (assuming that there are no other writes on {\tt x} in the same transaction).
%In figure (\ref{rr_eg:1}), the operation \texttt{print(x)} in the transaction should not print \texttt{1}, because $\texttt{print(x)}$ is not the latest write to \texttt{print(x)}.
These two properties can be verified easily (in a syntactic manner) on a given execution. Beyond these two properties, the various consistency criteria used in practice constrain only the last writes to each variable in each transaction and the reads that are not preceded by writes to the same variable in the same transaction.

Consistency criteria are formalized on an abstract view of an execution called~\emph{history}. A history includes only successful or committed transactions. In the context of databases, it is always assumed that the effect of aborted transactions should not be visible to other transactions, and therefore, they can be ignored. For instance, the \texttt{read(x)} in Figure~\ref{abort:1} should not return the value 1 written by the aborted transaction. The transactions are ordered according to a (partial) \emph{session order} $\so$ which represents ordering constraints imposed by the applications using the database. Most often, $\so$ is a union of sequences, each sequence being called a \emph{session}. We assume that the history includes a \emph{write-read} relation that identifies the transaction writing the value returned by each read in the execution. As mentioned before, such a relation can be extracted easily from executions where each value is written at most once. Since in practice, databases are data-independent~\cite{DBLP:conf/popl/Wolper86}, i.e., their behavior does not depend on the concrete values read or written in the transactions, any potential buggy behavior can be exposed in such executions. 

%Transactions and operations may be failed or aborted. The effects of the failed or aborted transactions and operations should not be visible. So we assume the transactions and operations are always successful or committed. If such transactions and operations exist, we discard them after checking there is no operation that read from failed or aborted transactions or operations. Therefore  In figure (\ref{abort:1}), the left transaction is aborted. So the operation \texttt{print(x)} in the right transaction should not output \texttt{0}. 


%We assume, in each transaction if there is no write before a read, it reads either from the last write of an another transaction or an initial value. Also when a transaction reads a value after a write, it reads from last write before that read. \ie if $\rd{\xvar}{\val} \in \tr_1$ reads from $\wrt{\xvar}{\val} \in \tr_2$, then for any other $\wrt{\xvar}{\_} \in \tr_2$ or $\tr_1$, $\tup{\wrt{\xvar}{\_}, \wrt{\xvar}{\val}} \in O_{\tr_2}$ or $\tup{\rd{\xvar}{\val}, \wrt{\xvar}{\_}} \in O_{\tr_1}$.
%
%Once we have a set of transactions, we can statically check if this is true for all transactions. Then, for all variables, we can ignore all the writes except the final one and all the reads after first write in a transaction. For simplicity, we assume that 



% TODO EXPLAIN THE FOLLOWING SIMPLIFICATIONS: WE DON'T CARE ABOUT MULTIPLE WRITES SINCE ANYWAY, ALL CRITERIA REQUIRE THAT ONLY THE LAST ONE IS VISIBLE. ALSO, ONCE A VARIABLE IS WRITTEN EVERY CRITERION REQUIRES THAT THE READ RETURNS THE INTERNALLY WRITTEN VALUE. ALL THESE THINGS CAN BE CHECKED EASILY IN A SYNTACTIC WAY ON HISTORIES.


%We say that a transaction $\tr$ writes value $\val$ a variable $\xvar$, denoted by $\tr\models \wrt{\xvar}{\val}$, whenever $\wrt{\xvar}{\val}$ is the last write o variable $\xvar$ of $\tr$, i.e., $\wrt{\xvar}{\val}\in \writeOp{\tr}$ and for any $\wrt{\xvar}{\val'}\in \writeOp{\tr}$ with $\val\neq\val'$, we have that $\tup{\wrt{\xvar}{\val'}, \wrt{\xvar}{\val}}\in \po$.
% \item $\tr \models \wrt[\id]{\xvar}{\val} \Leftrightarrow max_{\textsf{po}} \left( \left\{ write(x,\_) \in O \right\} \right) = write(x, n)$
% \item $(O, \textsf{po}) \models \texttt{read}(x, n) \Leftrightarrow min_{\textsf{po}} \left( \left\{ \_(x,\_) \in O \right\} \right) = read(x, n)$
% \item $(O, \textsf{po}) \in Write_{x} \Leftrightarrow \left\{ write(x,\_) \in O \right\} \neq \emptyset$



%When a read operation in a transaction reads a value from an operation from another transaction, it defines a \emph{write-read order} between them. Also, typically a transactional system allows its clients to group their transactions in one single session. This imposes a \emph{session order} on the transactions. We define a client-visible results of an execution of a set of sessions as a \emph{history}.

% TODO SAY THAT HISTORIES CONTAIN ONLY COMMITTED TRANSACTIONS FROM A CONCRETE EXECUTION, THE EFFECTS OF ABORTED TRANSACTIONS ARE NOT VISIBLE.

\begin{definition}
 A \emph{history} $\tup{T, \so, \wro}$ is a set of transactions $T$ along with a strict partial order $\so$ called \emph{session order}, and a 
 %surjective\footnote{That is, for all $\rd{\xvar}{\val}\in \readOp{T}$ there exists a transaction $\tr\in T$ such that $\tup{\tr,\rd{\xvar}{\val}}\in \wro$.} 
 relation $\wro\subseteq T\times \readOp{T}$ called \emph{write-read} relation, s.t. 
 \begin{itemize}
  \item the inverse of $\wro$ is a total function, and if $(\tr,\rd{\xvar}{\val})\in\wro$, then $\wrt{\xvar}{\val}\in\tr$, and
  \item $\so\cup\wro$ is acyclic.
 \end{itemize}
\end{definition}

% TODO SAY THAT INITIAL VALUES ARE ASSUMED TO BE WRITTEN BY A TRANSACTION ORDERED IN SO BEFORE ALL THE OTHER TRANSACTIONS.

%The transactions may try to read a variable, even before there is a write to it. In practice, the databases return a default initialized or null value for those reads. This situation can be thought as if the database wrote that value in an \emph{initialization} transaction, in which all variables are written with an initialized value. So we assume a history contains an initialization transaction. This initialization transaction precedes all other transactions by $\so$.

To simplify the technical exposition, we assume that every history includes a distinguished transaction writing the initial values of all variables. This transaction precedes all the other transactions in $\so$. We use $\hist$, $\hist_1$, $\hist_2$, $\ldots$ to range over histories. 

We say that the read operation $\rd{\xvar}{\val}$ reads value $\val$ from variable $\xvar$ written by $\tr$ when $(\tr,\rd{\xvar}{\val})\in\wro$. 
For a given variable $\xvar$, $\wro[\xvar]$ denotes the restriction of $\wro$ to reads of variable $\xvar$, \ie, $\wro[\xvar]=\wro\cap (T\times \{\rd{\xvar}{\val}\mid \val\in \Val\})$. Moreover, we extend the relations $\wro$ and $\wro[\xvar]$ to pairs of transactions as follows: $\tup{\tr_1,\tr_2}\in \wro$, resp., $\tup{\tr_1,\tr_2}\in \wro[\xvar]$, iff there exists a read operation $\rd{\xvar}{\val}\in \readOp{\tr_2}$ such that $\tup{\tr_1,\rd{\xvar}{\val}}\in \wro$, resp., $\tup{\tr_1,\rd{\xvar}{\val}}\in \wro[\xvar]$. We say that the transaction $\tr_1$ is \emph{read} by the transaction $\tr_2$ when $\tup{\tr_1,\tr_2}\in \wro$, and that it is \emph{read} when it is read by some transaction $\tr_2$. 
%

%A consistency model describes how a transactional system processes transactions. To formally define the behaviors, we will extend history with two relations.

%The definitions of the consistency criteria rely on a few basic notions of sequences and orders. We denote the \emph{prefix} of a sequence $\sigma$ up to and including an element $\alpha$ by $\sigma(\alpha)$, and the \emph{prefix} of a partial order $\pi$ up to and including $\alpha$ by $\pi(\alpha) = \set{\tup{ \alpha_1, \alpha_2 } \in \pi : \tup{\alpha_2, \alpha} \in \pi \text{ or } \alpha_2 = \alpha }$. A sequence $\sigma_1$ is called a \emph{subsequence} of another sequence $\sigma_2$, denoted by $\sigma_1\preceq \sigma_2$, when $\sigma_1$ is obtained from $\sigma_2$ by deleting elements. We extend the notion of subsequence to partial orders and say that an order $\pi_1$ is a \emph{suborder} of another order $\pi_2$ if $\pi_1\subseteq \pi_2$. For uniformity, we write $\pi_1 \preceq \pi_2$ when $\pi_1$ is a suborder of $\pi_2$. The subsequence of a sequence $\sigma$ that includes all the elements ordered between two elements $\alpha_1$ and $\alpha_2$ (including $\alpha_1$ and $\alpha_2$) is denoted by $\sigma[\alpha_1,\alpha_2]$.
%Also, for simplicity, we don’t make the distinction between sequences and total orders, reusing notions like prefix and subsequence in the context of total orders.
%
%\begin{definition}
% A \emph{linearization} $\ell=\tup{\lin,\vis}$ of a history $\tup{T, \so, \wro}$ is a total order $\lin$ on $T$, and a function $\vis$, mapping each transaction $\tr\in T$ to a subsequence of $\lin(\tr)$ including $\tr$.
%\end{definition}
%
%A consistency model is defined as the set of histories which admit linearizations satisfying certain \emph{consistency axioms}. The axioms rely on a few notations
%\begin{itemize}
% \item $\tr \models \wrt[\id]{\xvar}{\val} \Leftrightarrow max_{\textsf{po}} \left( \left\{ write(x,\_) \in O \right\} \right) = write(x, n)$
% \item $(O, \textsf{po}) \models \texttt{read}(x, n) \Leftrightarrow min_{\textsf{po}} \left( \left\{ \_(x,\_) \in O \right\} \right) = read(x, n)$
% \item $(O, \textsf{po}) \in Write_{x} \Leftrightarrow \left\{ write(x,\_) \in O \right\} \neq \emptyset$
%\end{itemize}


\subsection{Axiomatic Framework}

% from Table. \ref{weakconsistency:1}. Table. \ref{weakconsistency:2} lists the axioms of consistency models.

%!TEX root = Thesis.tex

\tikzset{transaction state/.style={draw=black!0}}


 \begin{figure}
   \resizebox{\textwidth}{!}{
   \footnotesize
  \begin{tabular}{|c|c|c|}
   \hline &  & \\
   \begin{subfigure}[t]{.3\textwidth}
    \centering
    \begin{tikzpicture}[->,>=stealth',shorten >=1pt,auto,node distance=1cm,
      semithick, transform shape]
     \node[transaction state, text=red] at (0,0)       (t_1)           {$\tr_1$};
     \node[transaction state, text=red, label={above:\textcolor{red}{$\writeVar{ }{\xvar}$}}] at (-0.5,1.5) (t_2) {$\tr_2$};
     \node[transaction state, text=red] at (2,0)       (o_1)           {$\alpha$};
     \node[transaction state] at (1.5,1.5) (o_2) {$\beta$};
     \path (t_1) edge[red] node {$\wro[\xvar]$} (o_1);
     % \path (t_2) edge[blue] node {$\CO$} (t_1);
     \path (t_2) edge node {$\wro$} (o_2);
     \path (o_2) edge node {$\po$} (o_1);
     \path (t_2) edge[left,double] node {$\co$} (t_1);
    \end{tikzpicture}
    \parbox{\textwidth}{
     $\forall \xvar,\ \forall \tr_1, \tr_2,\ \forall \alpha.\ \tr_1\neq \tr_2\ \land$
     
     \hspace{4mm}$\tup{\tr_1,\alpha}\in \wro[\xvar] \land \writeVar{\tr_2}{\xvar}\ \land$ 
     
     \hspace{9mm}$\tup{\tr_2,\alpha}\in\wro\circ\po$
     
     \hspace{14mm}$\implies \tup{\tr_2,\tr_1}\in\co$
    }
    %\end{align*}
    
    \caption{$\mathsf{Read\ Committed}$}
    \label{lock_rc_def}
   \end{subfigure}
   
          &     
   
   \begin{subfigure}[t]{.3\textwidth}
    \centering
    \begin{tikzpicture}[->,>=stealth',shorten >=1pt,auto,node distance=4cm,
      semithick, transform shape]
     \node[transaction state, text=red] at (0,0)       (t_1)           {$\tr_1$};
     \node[transaction state] at (2,0)       (t_3)           {$\tr_3$};
     \node[transaction state, text=red,label={above:\textcolor{red}{$\writeVar{ }{\xvar}$}}] at (-.5,1.5) (t_2) {$\tr_2$};
     \path (t_1) edge[red] node {$\wro[\xvar]$} (t_3);
     % \path (t_2) edge[blue] node {$\CO$} (t_1);
     \path (t_2) edge[bend left] node {$\wro \cup \so$} (t_3);
     \path (t_2) edge[left,double] node {$\co$} (t_1);
    \end{tikzpicture}
    \parbox{\textwidth}{
     $\forall \xvar,\ \forall \tr_1, \tr_2,\ \forall \tr_3.\ \tr_1\neq \tr_2\ \land$
     
     \hspace{4mm}$\tup{\tr_1,\tr_3}\in \wro[\xvar] \land \writeVar{\tr_2}{\xvar}\ \land$ 
     
     \hspace{9mm}$\tup{\tr_2,\tr_3}\in\wro\cup\so$
     
     \hspace{14mm}$\implies \tup{\tr_2,\tr_1}\in\co$
    }
    
    \caption{$\mathsf{Read\ Atomic}$}
    \label{ra_def}
   \end{subfigure}
   
   &
   
   \begin{subfigure}[t]{.3\textwidth}
    \centering
    \begin{tikzpicture}[->,>=stealth',shorten >=1pt,auto,node distance=4cm,
      semithick, transform shape]
     \node[transaction state, text=red] at (0,0)       (t_1)           {$\tr_1$};
     \node[transaction state] at (2,0)       (t_3)           {$\tr_3$};
     \node[transaction state, text=red,label={above:\textcolor{red}{$\writeVar{ }{\xvar}$}}] at (-.5,1.5) (t_2) {$\tr_2$};
     \path (t_1) edge[red] node {$\wro[\xvar]$} (t_3);
     % \path (t_2) edge[blue] node {$\CO$} (t_1);
     \path (t_2) edge[dashed, bend left] node {$(\wro \cup \so)^+$} (t_3);
     %   \path [->, decoration={snake}] (t_2) edge[decorate] node[auto] {F} (t_3);
     \path (t_2) edge[left,double] node {$\co$} (t_1);
    \end{tikzpicture}
    \parbox{\textwidth}{
     $\forall \xvar,\ \forall \tr_1, \tr_2,\ \forall \tr_3.\ \tr_1\neq \tr_2\ \land$
     
     \hspace{4mm}$\tup{\tr_1,\tr_3}\in \wro[\xvar] \land \writeVar{\tr_2}{\xvar}\ \land$ 
     
     \hspace{9mm}$\tup{\tr_2,\tr_3}\in(\wro\cup\so)^+$
     
     \hspace{14mm}$\implies \tup{\tr_2,\tr_1}\in\co$
    }
    
    \caption{$\mathsf{Causal}$}
    \label{cc_def}
   \end{subfigure}
   
   
   \\ \hline & & \\
    
   \begin{subfigure}[t]{.3\textwidth}
    \centering
    \begin{tikzpicture}[->,>=stealth',shorten >=1pt,auto,node distance=4cm,
      semithick, transform shape]
     \node[transaction state, text=red] at (0,0)       (t_1)           {$\tr_1$};
     \node[transaction state] at (2,0)       (t_3)           {$\tr_3$};
     \node[transaction state, text=red,label={above:\textcolor{red}{$\writeVar{ }{\xvar}$}}] at (-0.5,1.5) (t_2) {$\tr_2$};
     \node[transaction state] at (1.5,1.5) (t_4) {$\tr_4$};
     \path (t_1) edge[red] node {$\wro[\xvar]$} (t_3);
     % \path (t_2) edge[blue] node {$\CO$} (t_1);
     \path (t_2) edge node {$\co^*$} (t_4);
     \path (t_4) edge[left] node {$(\wro \cup \so)$} (t_3);
     \path (t_2) edge[left,double] node {$\co$} (t_1);
    \end{tikzpicture}
    \parbox{\textwidth}{
     $\forall \xvar,\ \forall \tr_1, \tr_2,\ \forall \tr_3.\ \tr_1\neq \tr_2\ \land$
     
     \hspace{4mm}$\tup{\tr_1,\tr_3}\in \wro[\xvar] \land \writeVar{\tr_2}{\xvar}\ \land$ 
     
     \hspace{9mm}$\tup{\tr_2,\tr_3}\in\co^*\circ\,(\wro\cup\so)$
     
     \hspace{14mm}$\implies \tup{\tr_2,\tr_1}\in\co$
    }
    
    \caption{$\mathsf{Prefix}$}
    \label{pre_def}
   \end{subfigure}
          
   
   &
   \begin{subfigure}[t]{.32\textwidth}
    \centering
    \begin{tikzpicture}[->,>=stealth',shorten >=1pt,auto,node distance=4cm,
      semithick, transform shape]
     \node[transaction state, text=red] at (0,0)       (t_1)           {$\tr_1$};
     \node[transaction state, label={below:$\writeVar{ }{\yvar}$}] at (2,0)       (t_3)           {$\tr_3$};
     \node[transaction state, text=red,label={above:\textcolor{red}{$\writeVar{ }{\xvar}$}}] at (-.5,1.5) (t_2) {$\tr_2$};
     \node[transaction state, label={above:{$\writeVar{}{\yvar}$}}] at (1.5,1.5) (t_4) {$\tr_4$};
     \path (t_1) edge[red] node {$\wro[\xvar]$} (t_3);
     % \path (t_2) edge[blue] node {$\CO$} (t_1);
     \path (t_2) edge node {$\co^*$} (t_4);
     \path (t_4) edge node {$\co$} (t_3);
     \path (t_2) edge[left,double] node {$\co$} (t_1);
    \end{tikzpicture}
    \parbox{\textwidth}{
     $\forall \xvar,\ \forall \tr_1, \tr_2,\ \forall \tr_3, \tr_4,\ \forall \yvar.\ \tr_1\neq \tr_2\ \land$
     
     \hspace{4mm}$\tup{\tr_1,\tr_3}\in \wro[\xvar] \land \writeVar{\tr_2}{\xvar}\ \land$ 
     
     \hspace{9mm}$\writeVar{\tr_3}{\yvar}\ \land \writeVar{\tr_4}{\yvar}\ \land$ 
     
     \hspace{12mm}$\tup{\tr_2,\tr_4}\in\co^*\ \land \tup{\tr_4,\tr_3}\in\co$
     
     \hspace{16mm}$\implies \tup{\tr_2,\tr_1}\in\co$
    }
    
    \caption{$\mathsf{Conflict}$}
    \label{confl_def}
   \end{subfigure}
          &     
   \begin{subfigure}[t]{.3\textwidth}
    \centering
    \begin{tikzpicture}[->,>=stealth',shorten >=1pt,auto,node distance=4cm,
      semithick, transform shape]
     \node[transaction state, text=red] at (0,0)       (t_1)           {$\tr_1$};
     \node[transaction state] at (2,0)       (t_3)           {$\tr_3$};
     \node[transaction state, text=red, label={above:\textcolor{red}{$\writeVar{ }{\xvar}$}}] at (-.5,1.5) (t_2) {$\tr_2$};
     \path (t_1) edge[red] node {$\wro[\xvar]$} (t_3);
     % \path (t_2) edge[blue] node {$\CO$} (t_1);
     \path (t_2) edge[bend left] node {$\CO$} (t_3);
     \path (t_2) edge[left,double] node {$\co$} (t_1);
    \end{tikzpicture}
    \parbox{\textwidth}{
     $\forall \xvar,\ \forall \tr_1, \tr_2,\ \forall \tr_3.\ \tr_1\neq \tr_2\ \land$
     
     \hspace{4mm}$\tup{\tr_1,\tr_3}\in \wro[\xvar] \land \writeVar{\tr_2}{\xvar}\ \land$ 
     
     \hspace{9mm}$\tup{\tr_2,\tr_3}\in\co$
     
     \hspace{14mm}$\implies \tup{\tr_2,\tr_1}\in\co$
    }
    
    \caption{$\mathsf{Serializability}$}
    \label{ser_def}
   \end{subfigure}
   \\ \hline
  \end{tabular}
  }
  \caption{Definitions of consistency axioms. The reflexive and transitive, resp., transitive, closure of a relation $rel$ is denoted by $rel^*$, resp., $rel^+$. Also, $\circ$ denotes the composition of two relations, i.e., $rel_1 \circ rel_2 = \{\tup{a, b} | \exists c. \tup{a, c} \in rel_1 \land \tup{c, b} \in rel_2\}$.}
  \label{consistency_defs}
 \end{figure}



% Practically, \textsc{Int} ensures each transaction takes one global snapshot of variables at the beginning. Then no other global changes affect that snapshot for any read or write local to that transaction. \textsc{Ext} ensures each transaction always observes the latest global snapshot that is visible to it.

We describe an axiomatic framework to characterize the set of histories satisfying a certain consistency criterion. The overarching principle is to say that a history satisfies a certain criterion if there exists a strict total order on its transactions, called \emph{commit order} and denoted by $\co$, which extends the write-read relation and the session order, and which satisfies certain properties. These properties are expressed by a set of axioms that relate the commit order with the session-order and the write-read relation in the history. 

%\begin{figure}
%  
%   \centering
%   \begin{subfigure}{.22\textwidth}
%  \resizebox{\textwidth}{!}{
%  \begin{tikzpicture}[->,>=stealth',shorten >=1pt,auto,node distance=3cm,
%    semithick, transform shape]
%    % \node[draw, rounded corners=2mm] (t1) at (0, 0) {\begin{tabular}{l} \texttt{x = 1;} \end{tabular}};
%   \node[draw, rounded corners=2mm] (t2) at (0, -.75) {\begin{tabular}{l} \texttt{x = 1;} \\ \texttt{y = 1};\end{tabular}};
%   \node[draw, rounded corners=2mm, minimum width=3.5cm, minimum height=2.5cm] (t3) at (3, -0.75) {};
%   \node[draw=black!50, rounded corners=2mm, dashed] (t3_1) at (3, 0) {\begin{tabular}{l} \texttt{print(y); // 1} \end{tabular}};
%   \node[draw=black!50, rounded corners=2mm, dashed] (t3_2) at (3, -1.5) {\begin{tabular}{l} \texttt{print(x); // 0} \end{tabular}};
%   % \path (t1) edge node {} (t3_2);
%   % \path (t2) edge node {} (t3_1);
%   % \path (t1) edge node {$\co$} (t2);
%   \path (t3_1) edge node {$\po$} (t3_2);
%  \end{tikzpicture}  
%  }
%    \caption{$\mathsf{Read\ Committed}$ violation.}
%    \label{rc_example:1}
%    
%\end{subfigure}
%\begin{subfigure}{.22\textwidth}
%\resizebox{\textwidth}{!}{
%\begin{tikzpicture}[->,>=stealth',shorten >=1pt,auto,node distance=3cm,
% semithick, transform shape]
% % \node[draw, rounded corners=2mm] (t1) at (0, 0) {\begin{tabular}{l} \texttt{x = 1;} \end{tabular}};
%\node[draw, rounded corners=2mm] (t2) at (0, -.75) {\begin{tabular}{l} \texttt{x = 1;} \end{tabular}};
%\node[draw, rounded corners=2mm, minimum width=3.5cm, minimum height=2.5cm] (t3) at (3, -0.75) {};
%\node[draw=black!50, rounded corners=2mm, dashed] (t3_1) at (3, -0) {\begin{tabular}{l} \texttt{print(x); // 0} \end{tabular}};
%\node[draw=black!50, rounded corners=2mm, dashed] (t3_2) at (3, -1.5) {\begin{tabular}{l} \texttt{print(x); // 1} \end{tabular}};
%% \path (t1) edge node {} (t3_1);
%% \path (t2) edge node {} (t3_2);
%% \path (t1) edge node {$\co$} (t2);
%\path (t3_1) edge node {$\po$} (t3_2);
%\end{tikzpicture}  
%}
% \caption{Repeatable Read violation.}
% \label{rr_example:1}
%\end{subfigure}
%\begin{subfigure}{.22\textwidth}
%\resizebox{\textwidth}{!}{
%\begin{tikzpicture}[->,>=stealth',shorten >=1pt,auto,node distance=3cm,
% semithick, transform shape]
% % \node[draw, rounded corners=2mm] (t1) at (0, 1.5) {\begin{tabular}{l} \texttt{x = 1;} \\ \texttt{y = 1;}\end{tabular}};
%\node[draw, rounded corners=2mm] (t2) at (-.85, -.75) {\begin{tabular}{l} \texttt{print(x); // 0} \\ \texttt{y = 1};\end{tabular}};
%\node[draw, rounded corners=2mm, minimum width=3.5cm, minimum height=2.5cm] (t3) at (3, -0.75) {};
%\node[draw=black!50, rounded corners=2mm, dashed] (t3_1) at (3, 0) {\begin{tabular}{l} \texttt{print(x); // 0} \end{tabular}};
%\node[draw=black!50, rounded corners=2mm, dashed] (t3_2) at (3, -1.5) {\begin{tabular}{l} \texttt{print(y); // 0} \end{tabular}};
%% \path (t1) edge node {} (t3);
%\path (t2) edge node {$\so$} (t3);
%% \path (t1) edge node {} (t2);
%\path (t3_1) edge node {$\po$} (t3_2);
%\end{tikzpicture}  
%}
% \caption{Read My Writes violation.}
% \label{rmw_example:1}
%\end{subfigure}
%\begin{subfigure}{.22\textwidth}
%\resizebox{\textwidth}{!}{
%\begin{tikzpicture}[->,>=stealth',shorten >=1pt,auto,node distance=3cm,
% semithick, transform shape]
% % \node[draw, rounded corners=2mm] (t1) at (0, 0) {\begin{tabular}{l} \texttt{x = 1;} \\ \texttt{y = 1;} \end{tabular}};
%\node[draw, rounded corners=2mm] (t2) at (0, -.75) {\begin{tabular}{l} \texttt{x = 1;} \\ \texttt{y = 1};\end{tabular}};
%\node[draw, rounded corners=2mm, minimum width=3.5cm, minimum height=2.5cm] (t3) at (3, -0.75) {};
%\node[draw=black!50, rounded corners=2mm, dashed] (t3_2) at (3, -1.5) {\begin{tabular}{l} \texttt{print(y); // 1} \end{tabular}};
%\node[draw=black!50, rounded corners=2mm, dashed] (t3_1) at (3, -0) {\begin{tabular}{l} \texttt{print(x); // 0} \end{tabular}};
%% \path (t1) edge node {} (t3_1);
%% \path (t2) edge node {} (t3_2);
%% \path (t1) edge node {$\co$} (t2);
%\path (t3_1) edge node {$\po$} (t3_2);
%\end{tikzpicture}  
%}
% \caption{Repeatable Read violation.}
% \label{ra_example:1}
%\end{subfigure}
%
%
%\begin{subfigure}{.22\textwidth}
%  \centering
%\resizebox{.65\textwidth}{!}{
%\begin{tikzpicture}[->,>=stealth',shorten >=1pt,auto,node distance=3cm,
% semithick, transform shape]
% % \node[draw, rounded corners=2mm] (t1) at (0, 1.5) {\begin{tabular}{l} \texttt{x = 1;} \\ \texttt{y = 1;}\end{tabular}};
%\node[draw, rounded corners=2mm] (t2) at (1.5, 1.5) {\begin{tabular}{l} \texttt{print(x); // 0} \\ \texttt{x = 1;} \\ \texttt{y = 1;} \end{tabular}};
%\node[draw, rounded corners=2mm] (t3) at (1.5, 0) {\begin{tabular}{l} \texttt{print(x); // 1} \\ \texttt{print(y); // 0} \end{tabular}};
%% \path (t1) edge node {} (t3);
%% \path (t2) edge node {$\so$} (t3);
%% \path (t1) edge node {} (t2);
%% \path (t3_1) edge node {$\po$} (t3_2);
%\end{tikzpicture}  
%}
% \caption{$\mathsf{Causal}$ violation.}
% \label{cc_example:1}
%\end{subfigure}
%\begin{subfigure}{.22\textwidth}
%\resizebox{\textwidth}{!}{
%\begin{tikzpicture}[->,>=stealth',shorten >=1pt,auto,node distance=3cm,
% semithick, transform shape]
% % \node[draw, rounded corners=2mm] (t1) at (0, 0) {\begin{tabular}{l} \texttt{x = 1;} \\ \texttt{y = 1;}\end{tabular}};
% \node[draw, rounded corners=2mm] (t2) at (-1.7, -1.5) {\begin{tabular}{l} \texttt{print(x); // 0} \\ \texttt{x = 1;} \end{tabular}};
%\node[draw, rounded corners=2mm] (t3) at (1.7, -1.5) {\begin{tabular}{l} \texttt{print(y); // 0} \\ \texttt{y = 1;} \end{tabular}};
%\node[draw, rounded corners=2mm] (t4) at (-1.7, -3) {\begin{tabular}{l} \texttt{print(x); // 1} \\ \texttt{print(y); // 0} \end{tabular}};
%\node[draw, rounded corners=2mm] (t5) at (1.7, -3) {\begin{tabular}{l} \texttt{print(y); // 1} \\ \texttt{print(x); // 0} \end{tabular}};
%% \node[draw, rounded corners=2mm] (t3) at (1.5, 0) {\begin{tabular}{l} \texttt{print(x); // 2} \\ \texttt{print(y); // 1} \end{tabular}};
%% \path (t1) edge node {} (t3);
%% \path (t2) edge node {$\so$} (t3);
%% \path (t1) edge node {} (t2);
%% \path (t3_1) edge node {$\po$} (t3_2);
%\end{tikzpicture}  
%}
% \caption{$\mathsf{Prefix}$ violation.}
% \label{pre_example:1}
%\end{subfigure}
%
%
%\begin{subfigure}{.22\textwidth}
%\resizebox{\textwidth}{!}{
%\begin{tikzpicture}[->,>=stealth',shorten >=1pt,auto,node distance=3cm,
% semithick, transform shape]
% % \node[draw, rounded corners=2mm] (t1) at (0, 0) {\begin{tabular}{l} \texttt{x = 1;} \end{tabular}};
% \node[draw, rounded corners=2mm] (t2) at (-1.7, -1.2) {\begin{tabular}{l} \texttt{print(x); // 0} \\ \texttt{x = 1;} \end{tabular}};
% \node[draw, rounded corners=2mm] (t3) at (1.7, -1.2) {\begin{tabular}{l} \texttt{print(x); // 0} \\ \texttt{x = 1;} \end{tabular}};
% % \node[draw, rounded corners=2mm] (t3) at (0, -2.4) {\begin{tabular}{l} \texttt{print(x); // 2} \end{tabular}};
%% \node[draw, rounded corners=2mm] (t3) at (1.7, -1.5) {\begin{tabular}{l} \texttt{print(y); // 1} \\ \texttt{y = 2;} \end{tabular}};
%% \node[draw, rounded corners=2mm] (t3) at (1.5, 0) {\begin{tabular}{l} \texttt{print(x); // 2} \\ \texttt{print(y); // 1} \end{tabular}};
%% \path (t1) edge node {} (t3);
%% \path (t2) edge node {$\co$} (t3);
%% \path (t1) edge node {} (t2);
%% \path (t3_1) edge node {$\po$} (t3_2);
%\end{tikzpicture}  
%}
% \caption{$\mathsf{Conflict}$ violation.}
% \label{conf_example:1}
%\end{subfigure}
%\begin{subfigure}{.22\textwidth}
%\resizebox{\textwidth}{!}{
%\begin{tikzpicture}[->,>=stealth',shorten >=1pt,auto,node distance=3cm,
% semithick, transform shape]
% % \node[draw, rounded corners=2mm] (t1) at (0, 0) {\begin{tabular}{l} \texttt{x = 1;} \\ \texttt{y = 1;}\end{tabular}};
% \node[draw, rounded corners=2mm] (t2) at (-1.7, -1.5) {\begin{tabular}{l} \texttt{print(x); // 0} \\ \texttt{print(y); // 0} \\ \texttt{x = 1;} \end{tabular}};
% \node[draw, rounded corners=2mm] (t3) at (1.7, -1.5) {\begin{tabular}{l} \texttt{print(x); // 0} \\ \texttt{print(y); // 0} \\ \texttt{y = 1;} \end{tabular}};
%% \node[draw, rounded corners=2mm] (t3) at (1.7, -1.5) {\begin{tabular}{l} \texttt{print(y); // 1} \\ \texttt{y = 2;} \end{tabular}};
%% \node[draw, rounded corners=2mm] (t3) at (1.5, 0) {\begin{tabular}{l} \texttt{print(x); // 2} \\ \texttt{print(y); // 1} \end{tabular}};
%% \path (t1) edge node {} (t3);
%% \path (t2) edge node {$\so$} (t3);
%% \path (t1) edge node {} (t2);
%% \path (t3_1) edge node {$\po$} (t3_2);
%\end{tikzpicture}  
%}
% \caption{$\mathsf{Serializability}$ violation.}
% \label{ser_example:1}
%\end{subfigure}
%
%  \caption{Examples of histories. For readability, the $\wro$ relation is defined by the values written in comments with each {\tt read}.}
%  \label{counter_example:1}
%\end{figure}



\begin{figure}
  
   \centering
   \begin{subfigure}{.32\textwidth}
  \resizebox{\textwidth}{!}{
\begin{tikzpicture}[->,>=stealth',shorten >=1pt,auto,node distance=3cm,
    semithick, transform shape]
    \node[draw, rounded corners=2mm] (t1) at (0, 0) {\begin{tabular}{l} \texttt{x = 1;} \end{tabular}};
   \node[draw, rounded corners=2mm] (t2) at (0, -1.5) {\begin{tabular}{l} \texttt{x = 2;} \\ \texttt{y = 2};\end{tabular}};
   \node[draw, rounded corners=2mm, minimum width=3.5cm, minimum height=2.5cm] (t3) at (3, -0.75) {};
   \node (t3_1) at (3, 0) {\begin{tabular}{l} \texttt{read(y); // 2} \end{tabular}};
   \node (t3_2) at (3, -1.5) {\begin{tabular}{l} \texttt{read(x); // 1} \end{tabular}};
   % \path (t1) edge node {} (t3_2);
   % \path (t2) edge node {} (t3_1);
   \path (t1) edge node {$\co$} (t2);
   \path (t3_1) edge node {$\po$} (t3_2);
  \end{tikzpicture}  
    }
    \caption{$\mathsf{Read\ Committed}$ violation.}
    \label{rc_example:1}
    
\end{subfigure}
\begin{subfigure}{.32\textwidth}
\resizebox{\textwidth}{!}{
\begin{tikzpicture}[->,>=stealth',shorten >=1pt,auto,node distance=3cm,
 semithick, transform shape]
 \node[draw, rounded corners=2mm] (t1) at (0, 0) {\begin{tabular}{l} \texttt{x = 1;} \end{tabular}};
\node[draw, rounded corners=2mm] (t2) at (0, -1.5) {\begin{tabular}{l} \texttt{x = 2;} \end{tabular}};
\node[draw, rounded corners=2mm, minimum width=3.5cm, minimum height=2.5cm] (t3) at (3, -0.75) {};
\node (t3_1) at (3, -0) {\begin{tabular}{l} \texttt{read(x); // 1} \end{tabular}};
\node (t3_2) at (3, -1.5) {\begin{tabular}{l} \texttt{read(x); // 2} \end{tabular}};
% \path (t1) edge node {} (t3_1);
% \path (t2) edge node {} (t3_2);
\path (t1) edge node {$\co$} (t2);
\path (t3_1) edge node {$\po$} (t3_2);
\end{tikzpicture}  
}
 \caption{Repeatable Read violation.}
 \label{rr_example:1}
\end{subfigure}
\begin{subfigure}{.32\textwidth}
\resizebox{\textwidth}{!}{
\begin{tikzpicture}[->,>=stealth',shorten >=1pt,auto,node distance=3cm,
 semithick, transform shape]
 \node[draw, rounded corners=2mm] (t1) at (0, 1.5) {\begin{tabular}{l} \texttt{x = 1;} \\ \texttt{y = 1;}\end{tabular}};
\node[draw, rounded corners=2mm] (t2) at (3, 1.5) {\begin{tabular}{l} \texttt{read(x); // 1} \\ \texttt{y = 2};\end{tabular}};
\node[draw, rounded corners=2mm, minimum width=3.5cm, minimum height=2.5cm] (t3) at (3, -0.75) {};
\node (t3_1) at (3, 0) {\begin{tabular}{l} \texttt{read(x); // 1} \end{tabular}};
\node (t3_2) at (3, -1.5) {\begin{tabular}{l} \texttt{read(y); // 1} \end{tabular}};
% \path (t1) edge node {} (t3);
\path (t2) edge node {$\so$} (t3);
% \path (t1) edge node {} (t2);
\path (t3_1) edge node {$\po$} (t3_2);
\end{tikzpicture}  
}
 \caption{Read My Writes violation.}
 \label{rmw_example:1}
\end{subfigure}
\begin{subfigure}{.32\textwidth}
\resizebox{\textwidth}{!}{
\begin{tikzpicture}[->,>=stealth',shorten >=1pt,auto,node distance=3cm,
 semithick, transform shape]
 \node[draw, rounded corners=2mm] (t1) at (0, 0) {\begin{tabular}{l} \texttt{x = 1;} \\ \texttt{y = 1;} \end{tabular}};
\node[draw, rounded corners=2mm] (t2) at (0, -1.5) {\begin{tabular}{l} \texttt{x = 2;} \\ \texttt{y = 2};\end{tabular}};
\node[draw, rounded corners=2mm, minimum width=3.5cm, minimum height=2.5cm] (t3) at (3, -0.75) {};
\node (t3_2) at (3, -1.5) {\begin{tabular}{l} \texttt{read(y); // 2} \end{tabular}};
\node (t3_1) at (3, -0) {\begin{tabular}{l} \texttt{read(x); // 1} \end{tabular}};
% \path (t1) edge node {} (t3_1);
% \path (t2) edge node {} (t3_2);
% \path (t1) edge node {$\co$} (t2);
\path (t3_1) edge node {$\po$} (t3_2);
\end{tikzpicture}  
}
 \caption{Repeatable Read violation.}
 \label{ra_example:1}
\end{subfigure}
\begin{subfigure}{.32\textwidth}
\resizebox{\textwidth}{!}{
\begin{tikzpicture}[->,>=stealth',shorten >=1pt,auto,node distance=3cm,
 semithick, transform shape]
 \node[draw, rounded corners=2mm] (t1) at (0, 1.5) {\texttt{x = 1;}};
\node[draw, rounded corners=2mm] (t2) at (3, 1.5) {\begin{tabular}{l} \texttt{read(x); // 1} \\ \texttt{x = 2;} \end{tabular}};
\node[draw, rounded corners=2mm] (t3) at (3, 0) {\begin{tabular}{l} \texttt{read(x); // 1} \\ \texttt{read(y); // 1} \end{tabular}};
\node[draw, rounded corners=2mm] (t4) at (0, 0) {\begin{tabular}{l} \texttt{read(x); // 2} \\ \texttt{y = 1;} \end{tabular}};
% \path (t1) edge node {} (t3);
% \path (t2) edge node {$\so$} (t3);
% \path (t1) edge node {} (t2);
% \path (t3_1) edge node {$\po$} (t3_2);
\end{tikzpicture}  
}
 \caption{$\mathsf{Causal}$ violation.}
 \label{cc_example:1}
\end{subfigure}
\begin{subfigure}{.33\textwidth}
\resizebox{\textwidth}{!}{
\begin{tikzpicture}[->,>=stealth',shorten >=1pt,auto,node distance=3cm,
 semithick, transform shape]
 \node[draw, rounded corners=2mm] (t1) at (0, 0) {\begin{tabular}{l} \texttt{x = 1;} \\ \texttt{y = 1;}\end{tabular}};
 \node[draw, rounded corners=2mm] (t2) at (-1.7, -1.5) {\begin{tabular}{l} \texttt{read(x); // 1} \\ \texttt{x = 2;} \end{tabular}};
\node[draw, rounded corners=2mm] (t3) at (1.7, -1.5) {\begin{tabular}{l} \texttt{read(y); // 1} \\ \texttt{y = 2;} \end{tabular}};
\node[draw, rounded corners=2mm] (t4) at (-1.7, -3) {\begin{tabular}{l} \texttt{read(x); // 2} \\ \texttt{read(y); // 1} \end{tabular}};
\node[draw, rounded corners=2mm] (t5) at (1.7, -3) {\begin{tabular}{l} \texttt{read(y); // 2} \\ \texttt{read(x); // 1} \end{tabular}};
% \node[draw, rounded corners=2mm] (t3) at (1.5, 0) {\begin{tabular}{l} \texttt{read(x); // 2} \\ \texttt{read(y); // 1} \end{tabular}};
% \path (t1) edge node {} (t3);
% \path (t2) edge node {$\so$} (t3);
% \path (t1) edge node {} (t2);
% \path (t3_1) edge node {$\po$} (t3_2);
\end{tikzpicture}  
}
 \caption{$\mathsf{Prefix}$ violation.}
 \label{pre_example:1}
\end{subfigure}
\begin{subfigure}{.32\textwidth}
\resizebox{\textwidth}{!}{
\begin{tikzpicture}[->,>=stealth',shorten >=1pt,auto,node distance=3cm,
 semithick, transform shape]
 \node[draw, rounded corners=2mm] (t1) at (0, 0) {\begin{tabular}{l} \texttt{x = 1;} \end{tabular}};
 \node[draw, rounded corners=2mm] (t2) at (-1.5, -1.2) {\begin{tabular}{l} \texttt{read(x); // 1} \\ \texttt{x = 2;} \end{tabular}};
 \node[draw, rounded corners=2mm] (t3) at (1.5, -1.2) {\begin{tabular}{l} \texttt{read(x); // 1} \\ \texttt{x = 3;} \end{tabular}};
 % \node[draw, rounded corners=2mm] (t3) at (0, -2.4) {\begin{tabular}{l} \texttt{read(x); // 2} \end{tabular}};
% \node[draw, rounded corners=2mm] (t3) at (1.7, -1.5) {\begin{tabular}{l} \texttt{read(y); // 1} \\ \texttt{y = 2;} \end{tabular}};
% \node[draw, rounded corners=2mm] (t3) at (1.5, 0) {\begin{tabular}{l} \texttt{read(x); // 2} \\ \texttt{read(y); // 1} \end{tabular}};
% \path (t1) edge node {} (t3);
% \path (t2) edge node {$\co$} (t3);
% \path (t1) edge node {} (t2);
% \path (t3_1) edge node {$\po$} (t3_2);
\end{tikzpicture} 
}
 \caption{$\mathsf{Conflict}$ violation.}
 \label{conf_example:1}
\end{subfigure}
\begin{subfigure}{.32\textwidth}
\resizebox{\textwidth}{!}{
\begin{tikzpicture}[->,>=stealth',shorten >=1pt,auto,node distance=3cm,
 semithick, transform shape]
 \node[draw, rounded corners=2mm] (t1) at (0, 0) {\begin{tabular}{l} \texttt{x = 1;} \\ \texttt{y = 1;}\end{tabular}};
 \node[draw, rounded corners=2mm] (t2) at (-1.7, -1.5) {\begin{tabular}{l} \texttt{read(x); // 1} \\ \texttt{read(y); // 1} \\ \texttt{x = 2;} \end{tabular}};
 \node[draw, rounded corners=2mm] (t3) at (1.7, -1.5) {\begin{tabular}{l} \texttt{read(x); // 1} \\ \texttt{read(y); // 1} \\ \texttt{y = 2;} \end{tabular}};
% \node[draw, rounded corners=2mm] (t3) at (1.7, -1.5) {\begin{tabular}{l} \texttt{read(y); // 1} \\ \texttt{y = 2;} \end{tabular}};
% \node[draw, rounded corners=2mm] (t3) at (1.5, 0) {\begin{tabular}{l} \texttt{read(x); // 2} \\ \texttt{read(y); // 1} \end{tabular}};
% \path (t1) edge node {} (t3);
% \path (t2) edge node {$\so$} (t3);
% \path (t1) edge node {} (t2);
% \path (t3_1) edge node {$\po$} (t3_2);
\end{tikzpicture}  
}
 \caption{$\mathsf{Serializability}$ violation.}
 \label{ser_example:1}
\end{subfigure}

  \caption{Examples of histories used to explain the axioms in Figure~\ref{consistency_defs}. For readability, the $\wro$ relation is defined by the values written in comments with each {\tt read}.}
  \label{counter_example:1}
\vspace{-3mm}
\end{figure}

The axioms we use have a uniform shape: they define mandatory $\co$ predecessors $\tr_2$ of a transaction $\tr_1$ that is read in the history. For instance, the criterion called \textsc{Read Committed} (RC)~\cite{DBLP:conf/sigmod/BerensonBGMOO95} requires that every value read in the history was written by a committed transaction, and also, that the reads in the same transaction are ``monotonic'' in the sense that they do not return values that are older, w.r.t. the commit order, than other values read in the past\footnote{This monotonicity property corresponds to the fact that in the original formulation of \textsc{Read Committed}~\cite{DBLP:conf/sigmod/BerensonBGMOO95}, every write is guarded by the acquisition of a lock on the written variable, that is held until the end of the transaction.}. While the first condition holds for any history (because of the surjectivity of $\wro$), the second condition is expressed by the axiom $\mathsf{Read\ Committed}$ in Figure~\ref{lock_rc_def}. This axiom states that for any transaction $\tr_1$ writing a variable $\xvar$ that is read in a transaction $\tr$, the set of transactions $\tr_2$ writing $\xvar$ and read previously in the same transaction must precede $\tr_1$ in commit order. For instance, Figure~\ref{rc_example:1} shows a history and a (partial) commit order that does not satisfy this axiom because ${\tt read(x)}$ returns the value written in a transaction ``older'' than the transaction read in the previous ${\tt read(y)}$. An example of a history and commit order satisfying this axiom is given in Figure~\ref{rr_example:1}.

%TODO GIVE A POSITIVE AND A NEGATIVE EXAMPLE W.R.T. READ COMMITTED, AND EXPLAIN THE APPLICATION OF THIS AXIOM. THE EXAMPLES SHOULD BE "SYNTHETIC", WITH VARIABLES $\xvar$ and $\yvar$.
More precisely, the axioms are first-order formulas\footnote{These formulas are interpreted on tuples $\tup{\hist,\co}$ of a history $\hist$ and a commit order $\co$ on the transactions in $\hist$ as usual.} of the following form:
\begin{align*}
  & \forall \xvar,\ \forall \tr_1,\tr_2,\ \forall \alpha.\ \tr_1\neq \tr_2\land \tup{\tr_1,\alpha}\in \wro[\xvar] \land \writeVar{\tr_2}{\xvar} \land \phi(\tr_2,\alpha) \implies \tup{\tr_2,\tr_1}\in\co 
\end{align*}
where $\phi$ is a property relating $\tr_2$ and $\alpha$ (i.e., the read or the transaction reading from $\tr_1$) that varies from one axiom to another. Intuitively, this axiom schema states the following: in order for $\alpha$ to read specifically $t_1$'s write on $x$, it must be the case that every $t_2$ that also writes $x$ and satisfies $\phi(t_2,\alpha)$ was committed before $t_1$. Note that in all cases we consider, $\phi(t_2,\alpha)$ already ensures that $t_2$ is committed before the read $\alpha$, so this axiom schema ensures that $t_2$ is furthermore committed before $t_1$'s write.

The axioms used throughout the paper are given in Figure~\ref{consistency_defs}. The property $\phi$ relates $\tr_2$ and $\alpha$ using the write-read relation and the session order in the history, and the commit order. 
%The axioms are first-order formulas whose satisfaction on tuples $\tup{\hist,\co}$ of a history $\hist$ and a commit order $\co$ on the transactions in $\hist$ is defined as usual.

In the following, we explain the rest of the consistency criteria we consider and the axioms defining them. \textsc{Read Atomic} (RA)~\cite{DBLP:conf/concur/Cerone0G15} is a strengthening of \textsc{Read Committed} defined by the axiom $\mathsf{Read\ Atomic}$, which states that for any transaction $\tr_1$ writing a variable $\xvar$ that is read in a transaction $\tr_3$, the set of $\wro$ or $\so$ predecessors of $\tr_3$ writing $\xvar$ must precede $\tr_1$ in commit order. The case of $\wro$ predecessors corresponds to the Repeatable Read criterion in~\cite{DBLP:conf/sigmod/BerensonBGMOO95}, which requires that successive reads of the same variable in the same transaction return the same value, Figure~\ref{rr_example:1} showing a violation, and also that every read of a variable $\xvar$ in a transaction $\tr$ returns the value written by the maximal transaction $\tr'$ (w.r.t. the commit order) that is read by $\tr$, Figure~\ref{ra_example:1} showing a violation (for any commit order between the transactions on the left, either ${\tt read(x)}$ or ${\tt read(y)}$ will return a value not written by the maximal transaction). The case of $\so$ predecessors corresponds to the  ``read-my-writes'' guarantee~\cite{DBLP:conf/pdis/TerryDPSTW94} concerning sessions, which states that a transaction $\tr$ must observe previous writes in the same session. For instance, {\tt read(y)} returning 1 in Figure~\ref{rmw_example:1} shows that the last transaction on the right does not satisfy this guarantee: the transaction writing 1 to {\tt y} was already visible to that session before it wrote 2 to {\tt y}, and therefore the value 2 should have been read. $\mathsf{Read\ Atomic}$ requires that the $\so$ predecessor of the transaction reading {\tt y} be ordered in $\co$ before the transaction writing 1 to {\tt y}, which makes the union $\co\cup\wro$ cyclic.

The following lemma shows that for histories satisfying $\mathsf{Read\ Atomic}$, the inverse of $\wro[\xvar]$ extended to transactions is a total function (see Appendix~\ref{app:definitions} for the proof).

\begin{lemma}
 Let $\hist=\tup{T, \so, \wro}$ be a history. 
 If $\tup{\hist,\co}$ satisfies $\mathsf{Read\ Atomic}$, then %the extension of $\wro[\xvar]$ to transactions 
 for every transaction $\tr$ and two reads $\rd[\id_1]{\xvar}{\val_1},\rd[\id_2]{\xvar}{\val_2}\in \readOp{\tr}$, $\wro^{-1}(\rd[\id_1]{\xvar}{\val_1})=\wro^{-1}(\rd[\id_2]{\xvar}{\val_2})$ and $\val_1 = \val_2$.
\end{lemma}

\textsc{Causal Consistency} (CC)~\cite{DBLP:journals/cacm/Lamport78} is defined by the axiom $\mathsf{Causal}$, which states that for any transaction $\tr_1$ writing a variable $\xvar$ that is read in a transaction $\tr_3$, the set of $(\wro\cup \so)^+$ predecessors of $\tr_3$ writing $\xvar$ must precede $\tr_1$ in commit order ($(\wro\cup \so)^+$ is usually called the \emph{causal} order). A violation of this axiom can be found in Figure~\ref{cc_example:1}: the transaction $\tr_2$ writing 2 to {\tt x} is a $(\wro\cup \so)^+$ predecessor of the transaction $\tr_3$ reading 1 from {\tt x} because the transaction $\tr_4$, writing 1 to {\tt y}, reads {\tt x} from $\tr_2$ and $\tr_3$ reads {\tt y} from $\tr_4$. This implies that $\tr_2$ should precede in commit order the transaction $\tr_1$ writing 1 to {\tt x}, which again, is inconsistent with the write-read relation ($\tr_2$ reads from $\tr_1$).

\begin{table}[t]
 \caption{Consistency model definitions}
%  \vspace{2mm}
 \centering
 % \resizebox{\textwidth}{!}{
 \begin{tabular}{|l|l|}
  \hline
  \shortstack{Consistency model}   & Axioms                                   \\
  \hline
  \textsc{Read Committed} (RC)     & $\mathsf{Read\ Committed}$               \\
  \hline
  \textsc{Read Atomic} (RA)        & $\mathsf{Read\ Atomic}$                  \\
  \hline
  \textsc{Causal consistency} (CC) & $\mathsf{Causal}$                        \\
  \hline
  \textsc{Prefix consistency} (PC) & $\mathsf{Prefix}$                        \\
  \hline
  \textsc{Snapshot isolation} (SI) & $\mathsf{Prefix}\land \mathsf{Conflict}$ \\
  \hline
  \textsc{Serializability} (SER)   & $\mathsf{Serializability}$               \\
  \hline
  % \multicolumn{3}{|c|}{
  %  $\textbf{RA} \supset \textbf{CC} \supset \textbf{PC} \supset \textbf{SI} \supset \textbf{SER}$
  % }                                                                   \\
  % \hline
 \end{tabular}
 % }
 \label{weakconsistency:2}
% \vspace{-5mm}
\end{table}

\textsc{Prefix consistency} (PC)~\cite{DBLP:conf/ecoop/BurckhardtLPF15} is a strengthening of CC, which requires that every transaction observes a prefix of a commit order between all the transactions. With the intuition that the observed transactions are $\wro\cup\so$ predecessors, the axiom $\mathsf{Prefix}$ defining PC, states that for any transaction $\tr_1$ writing a variable $\xvar$ that is read in a transaction $\tr_3$, the set of $\co^*$ predecessors of transactions observed by $\tr_3$ writing $\xvar$ must precede $\tr_1$ in commit order (we use $\co^*$ to say that even the transactions observed by $\tr_3$ must precede $\tr_1$). This ensures the prefix property stated above. An example of a PC violation can be found in Figure~\ref{pre_example:1}: the two transactions on the bottom read from the three transactions on the top, but any serialization of those three transactions will imply that one of the combinations {\tt x=1}, {\tt y=2} or {\tt x=2}, {\tt y=1} cannot be produced at the end of a prefix in this serialization.

\textsc{Snapshot Isolation} (SI)~\cite{DBLP:conf/sigmod/BerensonBGMOO95} is a strengthening of PC that disallows two transactions to observe the same prefix of a commit order if they \emph{conflict}, i.e., write to a common variable. It is defined by the conjunction of $\mathsf{Prefix}$ and another axiom called $\mathsf{Conflict}$, which requires that for any transaction $\tr_1$ writing a variable $\xvar$ that is read in a transaction $\tr_3$, the set of $\co^*$ predecessors writing $\xvar$ of transactions conflicting with $\tr_3$ and before $\tr_3$ in commit order, must precede $\tr_1$ in commit order. Figure~\ref{conf_example:1} shows a $\mathsf{Conflict}$ violation.

Finally, \textsc{Serializability} (SER)~\cite{DBLP:journals/jacm/Papadimitriou79b} is defined by the axiom with the same name, which requires that for any transaction $\tr_1$ writing to a variable $\xvar$ that is read in a transaction $\tr_3$, the set of $\co$ predecessors of $\tr_3$ writing $\xvar$ must precede $\tr_1$ in commit order. This ensures that each transaction observes the effects of all the $\co$ predecessors. Figure~\ref{ser_example:1} shows a $\mathsf{Serializability}$ violation.

The next lemma states the relationship between these axioms (see Appendix~\ref{app:definitions} for the proof).

\begin{lemma}
 The following entailments hold:
 \begin{align*}
   & \mathsf{Causal} \implies \mathsf{Read\ Atomic}\implies \mathsf{Read\ Committed} \\
   & \mathsf{Prefix} \implies \mathsf{Causal}                                        \\
   & \mathsf{Serializability} \implies \mathsf{Prefix}\land \mathsf{Conflict}        
 \end{align*} 
 \label{axioms-rel}
\end{lemma}

\begin{definition}
 Given a set of axioms $X$ defining a criterion $C$ like in Table~\ref{weakconsistency:2}, a history $\hist=\tup{T, \so, \wro}$ \emph{satisfies} $C$ iff there exists a strict total order $\co$ such that $\wro\cup\so\subseteq \co$ and $\tup{h,\co}$ satisfies $X$.
 % Given a $\CO$(\textit{commit order}), a total order on $T$ which extends $\wro \cup \so$, we can define consistency axioms from table \ref{consistency_defs}. For each axiom, the situation in the table implies, $\Path{\tr_2}{\CO}{\tr_1}$.
 \label{axiom-criterion}
\end{definition}

Definition~\ref{axiom-criterion} and Lemma~\ref{axioms-rel} imply that each consistency criterion in Table~\ref{weakconsistency:2} is stronger than its predecessors (reading them from top to bottom), e.g., CC is stronger than RA and RC. This relation is known to be strict~\cite{DBLP:conf/concur/Cerone0G15}, e.g., RA is not stronger than CC. 
%stronger from top to bottom order. Infact each criteria is strictly stronger than its previously weaker criteria, \ie \textsf{Snapshot Isolation} does not imply \textsf{Serializability}. 
 % (see Appendix~\ref{app:gotsman}).

%\begin{table}
% \begin{tabular}{|c|c|}
%  \hline
%  Axioms          & Anomaly                                  \\
%  \hline
%  Causal          & Causal violation                         \\
%  \hline
%  Prefix          & Causal violation, Long fork              \\
%  \hline
%  Conflict        & Lost update                              \\
%  \hline
%  Serializability & \shortstack{Causal violation, Long fork, \\ Lost update, Write Skew}\\
%  \hline 
% \end{tabular}
% \caption{Axioms and anomalies}
% \label{axiom_anomaly}
%\end{table}

%\begin{definition}
% We define consistency criteria using the definitions in table \ref{weakconsistency:2}.
%\end{definition}



% For simplicity, in all the following definitions, when we say, there exists a $\CO$\textit{(commit order)}, we will implicitly mean, $\CO$ is a total order on $T$ which extends $\WR \cup \SO$.
% 
% \begin{definition}
%  By our definition of history, history is Read committed.
% \end{definition}
% 
% 
% \begin{definition}
%  A history is Locking Read Committed, if there exists a $\CO$, such that for any variables $x,y$ and transactions $\tr_1$, $\tr_2$ and two operations $\op_1$, $\op_2$ in any situation illustrated in figure \ref{lock_rc_def}, $\Path{\tr_2}{\CO}{\tr_1}$ always.
% \end{definition}
% 
% \begin{definition}
%  A history is Read atomic, resp., Causal consistency, Prefix consistency, Conflict consistency, Serializability, if there exists a $\CO$, such that for any variable $\xvar$ and transactions $\tr_1$, $\tr_2$, $\tr_3$ in the resp. situations illustrated in figure \ref{consistency_defs}, $\Path{\tr_2}{\CO}{\tr_1}$ always.
% \end{definition}

% \begin{definition}
%  The sets of histories and schedules allowed by a particular weak model - `wm' can be defined as:
% \end{definition}
% 
% \begin{itemize}
%  \item $\textsf{Sche}_{wm} = \left\{ \mathcal{S} \mid S \models \text{axioms of } wm \right\}$
%  \item $\textsf{Hist}_{wm} = \left\{ \mathcal{H} \mid \exists \VIS, \CO, (\mathcal{H}, VIS, CO)\right\}$
% \end{itemize}
% 
% So to verify a history for a consistency model, it suffices to find a schedule, in particular, \VIS, \CO for that history, which satisfies the axioms for that consistency model.


% \begin{table*}
%  \centering
%  \resizebox{\textwidth}{!}{
%   \begin{tabular}{|c|l|l|c|c|c|c|c|}
%    \hline
%    $\Phi$       & \shortstack{Consistency model                                                                                                                            \\ (strong session)} & Axioms & \shortstack{Fractured \\ reads} & \shortstack{Causality \\ violation} & \shortstack{Lost \\ update} & \shortstack{Long \\ fork} & \shortstack{Write \\ skew} \\
%    \hline
%    \textbf{RA}  & Read atomic                   & \textsc{Int} $\wedge$ \textsc{Ext} $\wedge$ \textsc{Session} & \ding{55} & \ding{51} & \ding{51} & \ding{51} & \ding{51} \\
%    \hline
%    \textbf{CC}  & Causal consistency            & \textbf{RA} $\wedge$ \textsc{TransVis}                       & \ding{55} & \ding{55} & \ding{51} & \ding{51} & \ding{51} \\
%    \hline
%    \textbf{PC}  & Prefix consistency            & \textbf{RA} $\wedge$  \textsc{Prefix}                        & \ding{55} & \ding{55} & \ding{51} & \ding{55} & \ding{51} \\
%    \hline
%    \textbf{SI}  & Snapshot isolation            & \textbf{PC} $\wedge$ \textsc{NoConflict}                     & \ding{55} & \ding{55} & \ding{55} & \ding{55} & \ding{51} \\
%    \hline
%    \textbf{SER} & Serializability               & \textbf{RA} $\wedge$ \textsc{TotalVis}                       & \ding{55} & \ding{55} & \ding{55} & \ding{55} & \ding{55} \\
%    \hline
%    \multicolumn{8}{|c|}{
%     $\textbf{RA} \supset \textbf{CC} \supset \textbf{PC} \supset \textbf{SI} \supset \textbf{SER}$
%    }                                                                                                                                                                       \\
%    \hline
%   \end{tabular}
%  }
%  \caption{Consistency model definitions, anomalies and relationship}
%  \label{weakconsistency:2}
% \end{table*}


%$T \xrightarrow{\VIS} S$ means all the writes of $T$ are visible to $S$. $T \xrightarrow{\CO} S$ means all the writes of $T$ are committed before $S$ committed all its writes. $\VIS \subseteq \CO$ ensures a transaction $T$ can be visible to another transaction $S$, if only $T$ is committed before $S$.
