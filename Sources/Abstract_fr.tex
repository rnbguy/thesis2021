%!TEX root = ../Thesis.tex

\begin{center}
  \textsc{Résumé}
\end{center}
%
\noindent
%

% \vfill

À mesure que \emph{l'internet} devient moins cher et plus rapide, les systèmes et les applications logicielles distribués deviennent de plus en plus omniprésents.
Aujourd'hui, ils sont à la base d'un très grand nombre de services en ligne tels que les banques, le commerce électronique, les réseaux sociaux, etc.
Au fur et à mesure que la popularité de ces logiciels augmente, il est très important qu'ils garantissent des niveaux élevés de fiabilité et de sécurité.

Les logiciels distribués modernes sont centrés sur l'utilisation de systèmes de stockage à grande échelle pour stocker et manipuler des données.
Pour assurer la persistance et la disponibilité des données en présence de pannes, ces systèmes maintiennent les données
en plusieurs copies stockées sur différents nœuds du réseau. Pour des raisons de performances, ces copies
peuvent diverger (temporairement), une instance de la soi-disant \emph{cohérence faible},
ce qui rend la sémantique des accès concurrents aux données très complexe.

Au cours des dernières années, de nombreuses solutions pour implémenter des systèmes de stockage à \emph{cohérence faible} ont été proposées.
Ces implémentations sont le plus souvent très complexes et sujettes aux erreurs.
Les niveaux spécifiques de cohérence faible qu'ils assurent ne sont le plus souvent décrits que de manière informelle, ce qui rend difficile le raisonnement sur leurs correction.
De plus, dans de nombreux cas, il existe des écarts importants entre les garanties mentionnées dans leur documentation et les garanties qu'elles fournissent réellement.

L'objectif de cette thèse est de proposer des techniques algorithmiques pour le \emph{teste automatisé} de systèmes distribués à cohérence faible par rapport à des \emph{spécifications formelles}. Nous étudions une classe importante de types de données distribués, appelés \emph{types de données répliqués sans conflit} (\emph{CRDT}), qui inclut de nombreuses variantes comme des registres, des ensembles, des tableaux, etc., et des \emph{systèmes (bases de données) transactionnels}, qui permettent des calculs sur des données isolés des autres calculs concurrents et tolérants aux pannes. Nous introduisons des spécifications formelles pour de tels systèmes et nous étudions la complexité asymptotique de la vérification de la correction d'une exécution donnée par rapport à ces spécifications. Nous étudions également le problème du teste des applications qui s'exécutent sur des systèmes transactionnels à cohérence faible, en introduisant un système de stockage en mémoire qui simule les comportements de ces systèmes par rapport à leurs spécifications formelles.

%\emph{Internet} est devenu une commodité dans la vie d'aujourd'hui. Grâce aux nouvelles technologies de télécommunication de pointe, l'accès à Internet devient de plus en plus rapide et bon marché. Naturellement, les solutions distribuées sont de plus en plus répandues pour résoudre les problèmes du monde réel. 
%
%Bien que les systèmes distribués aient été étudiés dans le passé, on assiste depuis peu à une multiplication des bases de données distribuées et des applications spécialisées dans différents aspects des configurations distribuées. En raison de ces spécialisations, ces implémentations sont souvent compliquées, sujettes aux erreurs et très différentes les unes des autres. Tester tous ces systèmes différents avec leurs critères correspondants représente un grand défi.
%
%Dans cette thèse, nous explorons les moyens automatisés pour tester ces systèmes distribués et aider les développeurs à ne pas introduire de bogues lors du développement d'applications distribuées et du choix du meilleur magasin de données sous-jacent. Nous étudions les problèmes de test concernant les \emph{conflict-free replicated data types} (CRDTs) et \emph{systèmes transactionnels} correspondant à différents modèles de consistance faible ou niveaux d'isolation.
%
%Les CRDTs comme les compteurs, les drapeaux, les ensembles, les registres et les tableaux fournissent une haute disponibilité et une tolérance de partition. Ils utilisent des mécanismes non triviaux pour résoudre les effets des mises à jour concurrentes des données répliquées. Naturellement, ces objets affaiblissent leurs garanties de cohérence pour atteindre la disponibilité et la tolérance de partition, et diverses notions de \emph{faible cohérence} capturent ces garanties.
%
%Les transactions simplifient la programmation concurrente en permettant des calculs sur des données partagées qui sont isolés des autres calculs concurrents et résistent aux défaillances. Les bases de données modernes fournissent différents modèles de cohérence ou niveaux d'isolation pour les transactions, correspondant à différents compromis entre cohérence et disponibilité.
%
%Cette thèse étudie la tractabilité de la vérification de la conformité des systèmes CRDT et transactionnels pour les modèles de cohérence correspondants. Dans les deux cas, pour capturer les garanties avec précision et faciliter le raisonnement symbolique, nous proposons de nouvelles caractérisations logiques. En développant de nouvelles réductions à partir de problèmes de satisfiabilité propositionnelle, et de nouveaux algorithmes de vérification de la cohérence, nous découvrons des résultats positifs et négatifs. Nous démontrons également que la tractabilité pour les cas difficiles peut être rachetée si un certain paramètre est limité, généralement le nombre de sessions ou de répliques dans le réseau.
%
%Enfin, nous présentons \tool{}, un système de stockage fictif en mémoire permettant de tester les applications basées sur des datastores distribués. En maintenant un journal des opérations passées et en utilisant les caractérisations formelles précédentes, il calcule toutes les valeurs de retour cohérentes possibles pour chaque opération \textrm{Read} au moment de l'exécution. \tool{} renvoie ensuite une valeur choisie uniformément dans cet ensemble. \tool{} supporte une interface clé-valeur ainsi que les requêtes SQL sous plusieurs modèles de cohérence. Nous montrons que \tool{} fournit une bonne couverture des comportements faibles possibles, qui est complète à la limite. Nous testons une variété d'applications pour les assertions qui échouent seulement sous une isolation faible. \tool{} peut violer chacune de ces assertions en un petit nombre de tentatives.
%
%En conclusion, nous développons de nouveaux cadres pour le test automatique d'un ensemble de systèmes distribués. Nous fournissons des algorithmes pour effectuer des tests automatisés sur ces systèmes et étudions leurs efficacités. Ce travail fournit un cadre et une direction pour des travaux futurs dans des domaines distribués similaires.

\medskip
\noindent
\textbf{Mots-clés:} Méthodes formelles, Concurrence, Systèmes distribués, Bases de données, Teste automatisé, Cohérence faible, Types de données répliqués, Transactions, Complexité.
%Système distribué, Stockage répliqué, Stockage distribué, Base de données, Système transactionnel, Application distribuée, Niveau d'isolement, Cohérence faible, Tests aléatoires, Tests unitaires, Couverture des tests, Complexité
