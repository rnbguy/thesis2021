%!TEX root = Thesis.tex

\section{Programming Language}

\begin{figure}
\small
\begin{align*}
\key\in \Keys\quad \xvar\in\Vars\quad \tab\in\Tables\quad \vec{c},\vec{c_1},\vec{c_2}\in \Columns^*
\end{align*}
\begin{align*}
\mathsf{Prog} &  \eqdef  \mathsf{Sess} \ \mid\  \mathsf{Sess}\,||\,\mathsf{Prog} \\
\mathsf{Sess} & \eqdef  \mathsf{Trans} \ \mid\  \mathsf{Trans}; \mathsf{Sess} \\
\mathsf{Trans} & \eqdef  \ibegin; \mathsf{Body}; \icommit \\
\mathsf{Body} & \eqdef  \mathsf{Instr} \ \mid\  \mathsf{Instr}; \mathsf{Body} \\
\mathsf{Instr} & \eqdef  \mathsf{InstrKV} \ \mid\  \mathsf{InstrSQL}\ \mid\  x := e \mid\ \iif{\phi(\vec{x})}{\mathsf{Instr}} \\
\mathsf{InstrKV} & \eqdef \xvar := \iread(\key)  \ \mid\  \iwrite(\key,\xvar) \\
\mathsf{InstrSQL} & \eqdef  \iselect{\vec{c_1}}{\xvar}{\tab}{\phi(\vec{c_2})} \ \mid\ \\
& \hspace{5mm} \iinsert{\tab}{\vec{x}} \ \mid\ \\
& \hspace{5mm} \idelete{\tab}{\phi(\vec{c})} \ \mid\ \\
& \hspace{5mm} \iupdate{\tab}{\vec{c_1}=\vec{x}}{\phi(\vec{c_2})} 
%\mathsf{Local} & \eqdef  x := e
\end{align*}
\vspace{-6mm}
\caption{Program syntax. The set of all keys is denoted by $\Keys$, $\Vars$ denotes the set of local variables, $\Tables$ the set of table names, and $\Columns$ the set of column names.
We use $\phi$ to denote Boolean expressions, and $e$ to denote expressions interpreted as values. We use $\vec{\cdot}$ to denote vectors of elements.}
\label{fig:syntax}
\vspace{-4mm}
\end{figure}

Figure~\ref{fig:syntax} lists the definition of two simple programming languages that we use to represent applications running on top of Key-Value or SQL stores, respectively. A program is a set of \emph{sessions} running in parallel, each session being composed of a sequence of \emph{transactions}. Each transaction is delimited by $\ibegin$ and $\icommit$ instructions\footnote{For simplicity, we assume that all the transactions in the program commit. Aborted transactions can be ignored when reasoning about safety because their effects should be invisible to other transactions.}, and its body contains instructions that access the store, and manipulate a set of local variables $\Vars$ ranged over using $\xvar$, $\yvar$, $\ldots$.

In case of a program running on top of a Key-Value store, the instructions can be reading the value of a key and storing it to a local variable $\xvar$ ($\xvar := \iread(\key)$) , writing the value of a local variable $\xvar$ to a key ($\iwrite(\key,\xvar)$), or an assignment to a local variable $\xvar$. The set of values of keys or local variables is denoted by $\Vals$. Assignments to local variables use expressions interpreted as values whose syntax is left unspecified. Each of these instructions can be guarded by a Boolean condition $\phi(\vec{x})$ over a set of local variables $\vec{x}$ (their syntax is not important). Other constructs like $\mathtt{while}$ loops can be defined in a similar way. Let $\KVProgs$ denote the set of programs where a transaction body can contain only such instructions.

For programs running on top of SQL stores, the instructions include simplified
versions of standard SQL instructions and assignments to local variables. These
programs run in the context of a \emph{database schema} which is a (partial)
function $\DBschema:\Tables\rightharpoonup 2^\Columns$ mapping table names in
$\Tables$ to sets of column names in $\Columns$. The SQL store is an
\emph{instance} of a database schema $\DBschema$, i.e., a function $\DBinst:
\mathsf{dom}(\DBschema)\rightarrow 2^{\Rows}$ mapping each table $\tab$ in the
domain of $\DBschema$ to a set of \emph{rows} of $\tab$, i.e., functions $r:\DBschema(\tab)\rightarrow\Vals$. We use $\Rows$ to denote the set of all rows.
The $\mathtt{SELECT}$ instruction retrieves the columns $\vec{c_1}$ from the set of rows of $\tab$ that satisfy $\phi(\vec{c_2})$ ($\vec{c_2}$ denotes the set of columns used in this Boolean expression), and stores them into a variable $\xvar$. $\mathtt{INSERT}$ adds a new row to $\tab$ with values $\vec{x}$, and $\mathtt{DELETE}$ deletes all rows from $\tab$ that satisfy a condition $\phi(\vec{c})$. The $\mathtt{UPDATE}$ instruction assigns the columns $\vec{c_1}$ of all rows of $\tab$ that satisfy $\phi(\vec{c_2})$ with values in $\vec{x}$.
Let $\SQLProgs$ denote the set of programs where a transaction body can contain only such instructions.
