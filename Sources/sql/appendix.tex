%!TEX root = Thesis.tex
\section{Correctness of the Operational Semantics}

\begin{figure} [t]
\small
  \centering
  \begin{mathpar}
    \inferrule[spawn*]{\tr \mbox{ fresh}\quad \mathsf{P}(j) = \ibegin; \mathsf{Body}; \icommit; \mathsf{S} \quad \vec{\mathsf{B}}(j) = \epsilon}{
      \hist,\vec{\gamma},\vec{\mathsf{B}},\mathsf{P}
      \Rightarrow
      \hist \oplus_j \tup{\tr,\emptyset,\emptyset},\vec{\gamma}[j\mapsto \emptyset],\vec{\mathsf{B}}[j\mapsto \mathsf{Body}],\mathsf{P}[j\mapsto \mathsf{S}]
    } 

    \inferrule[if-true]{\varphi(\vec{x})[x\mapsto \vec{\gamma}(j)(x): x\in\vec{x}]\mbox{ true} \\
    \vec{\mathsf{B}}(j) = \iif{\phi(\vec{x})}{\mathsf{Instr}};\mathsf{B}
    }{
      \hist,\vec{\gamma},\vec{\mathsf{B}}, \mathsf{P}
      \Rightarrow
      \hist,\vec{\gamma},\vec{\mathsf{B}}[j\mapsto \mathsf{Instr};\mathsf{B}],\mathsf{P}
    } 

    \inferrule[if-false]{\varphi(\vec{x})[x\mapsto \vec{\gamma}(j)(x): x\in\vec{x}]\mbox{ false} \\
    \vec{\mathsf{B}}(j) = \iif{\phi(\vec{x})}{\mathsf{Instr}};\mathsf{B}
    }{
      \hist,\vec{\gamma},\vec{\mathsf{B}}, \mathsf{P}
      \Rightarrow
      \hist,\vec{\gamma},\vec{\mathsf{B}}[j\mapsto \mathsf{B}],\mathsf{P}
    } 

    \inferrule[write]{v = \vec{\gamma}(j)(x)\quad \id\mbox{ fresh} \quad 
    \vec{\mathsf{B}}(j) = \iwrite(\key,\xvar);\mathsf{B}
    }{
      \hist,\vec{\gamma},\vec{\mathsf{B}}, \mathsf{P}
      \Rightarrow
      \hist \oplus_j \wrt[\id]{\key}{\val},\vec{\gamma},\vec{\mathsf{B}}[j\mapsto \mathsf{B}], \mathsf{P}
    } 

    \inferrule[read-local]{
    \wrt{\key}{\val}\mbox{ is the last write on $\key$ in $\tr$}\\
    \id\mbox{ fresh } \\
    \vec{\mathsf{B}}(j) = \xvar := \iread(\key);\mathsf{B}
    }{
      \hist,\vec{\gamma},\vec{\mathsf{B}}, \mathsf{P}
      \Rightarrow
      \hist \oplus_j \rd[\id]{\key}{\val},\vec{\gamma}[(j,\xvar)\mapsto \val],\vec{\mathsf{B}}[j\mapsto \mathsf{B}],\mathsf{P}
    } 

    \inferrule[read-extern*]{
    \vec{\mathsf{B}}(j) = \xvar := \iread(\key);\mathsf{B} \\
    \hist=(T,\so,\wro) \\
    \tr \mbox{ is the id of the last transaction log in $\so(j)$} \\
    \wrt{\key}{\val}\in\writeOp{\tr'}\mbox{ with $\tr'\in \transC{\hist,\vec{\mathsf{B}}}$ and $\tr\neq \tr'$} \\
    \id\mbox{ fresh }\\
    \hist' = (\hist \oplus_j \rd[\id]{\key}{\val}) \oplus \wro(\tr',\rd[i]{\key}{\val}) }{
      \hist,\vec{\gamma},\vec{\mathsf{B}}, \mathsf{P}
      \Rightarrow
      \hist',\vec{\gamma}[(j,\xvar)\mapsto \val],\vec{\mathsf{B}}[j\mapsto \mathsf{B}],\mathsf{P}
    } 
    
  \end{mathpar}
% \vspace{-5mm}
  \caption{A baseline operational semantics for $\KVProgs$ programs. Above, $\transC{\hist,\vec{\mathsf{B}}}$ denotes the set of transaction logs in $\hist$ that excludes those corresponding to live transactions, i.e., transaction logs $\tr''\in T$ such that $\tr''$ is the last transaction log in some $\so(j')$ and $\vec{B}(j')\neq\epsilon$.}
  \label{fig:op:sem:baseline:complete}
\end{figure}

This section provides more details about the proof of correctness for our operational semantics defined in Figure~\ref{fig:op:sem}. The complete definition of the baseline semantics is given in Figure~\ref{fig:op:sem:baseline:complete}.

The notion of prefix of a tuple $\{\hist_2,\co_2\}$ is formally defined as follows. For a relation $R\subseteq A\times B$, the restriction of $R$ to $A'\times B'$, denoted by $R\downarrow A'\times B'$, is defined by $\{(a,b): (a,b)\in R, a\in A', b\in B'\}$.
For $\hist_1=\tup{T_1, \so_1, \wro_1}$ and $\hist_2=\tup{T_2, \so_2, \wro_2}$, $\tup{\hist_1,\co_1}$ is a \emph{prefix} of $\tup{\hist_2,\co_2}$, denoted by $\tup{\hist_1,\co_1}\leq \tup{\hist_2,\co_2}$, iff $\tup{\hist_1,\co_1}\leq \tup{\hist_2,\co_2}$ iff $T_1=T_1'\cup\{\tup{t,O,\po}\}$, $T_2=T_2'\cup\{\tup{t,O',\po'}\}$, $T_1'\subseteq T_2'$, $O\subseteq O'$, $\po = \po' \downarrow O\times O$, $\so_1 = \so_2 \downarrow T_1\times T_1$, $\wro_1= \wro_2\downarrow T_1\times \readOp{T_1}$, and $\co_1= \co_2\downarrow T_1\times T_1$.

Then, a property $\phi(\tr_2,\alpha)$ used to define an axiom like in (\ref{eq:axiom}), is called \emph{monotonic} iff 
for every  $\tup{\hist_1,\co_1}\leq \tup{\hist_2,\co_2}$, 
\begin{align*}
\forall \tr_2, \forall \alpha. \tup{\hist_1,\co_1}\models \phi(\tr_2,\alpha) \Rightarrow  \tup{\hist_2,\co_2}\models \phi(\tr_2,\alpha).
\end{align*}

\begin{lemma}\label{lem:prefix}
For any monotonic axiom $X$, if $\tup{\hist_1,\co_1}\leq \tup{\hist_2,\co_2}$, then
\begin{align*}
\tup{\hist_2,\co_2}\mbox{ satisfies } X \Rightarrow \tup{\hist_1,\co_1}\mbox{ satisfies } X
\end{align*}
\end{lemma}
\begin{proof}(Sketch)
Given a monotonic axiom, the number of instantiations of $\forall k$, $\forall t_1$, $\forall t_2$, and $\forall \alpha$ from (\ref{eq:axiom}) that satisfy the left-hand side of the entailement in the context of $\tup{\hist_1,\co_1}$ is a subset of the same type of instantiations in the context of $\tup{\hist_2,\co_2}$. Therefore, the $\co$ constraints imposed in the context of $\tup{\hist_1,\co_1}$ (by the right-hand side of the entailement) are a subset of the $\co$ constraints imposed in the context of $\tup{\hist_2,\co_2}$. Since the latter are satisfied (because $\tup{\hist_2,\co_2}$  satisfies  $X$), the former are also satisfied and hence, $\tup{\hist_1,\co_1}$  satisfies  $X$.
\end{proof}

Lemma~\ref{lem:prefix} extends obviously to isolation levels defined as conjunctions of axioms (which is the case for all the isolation levels that we are aware of~\cite{DBLP:journals/pacmpl/BiswasE19}).

 \begin{theorem}
For any isolation level $I$ defined by a set of monotonic axioms,
\begin{align*}
\histOf[I]{\prog} = \{ h \in \histOf[*]{\prog}: h\mbox{ satisfies }I\}.
\end{align*}
 \end{theorem}
 \begin{proof}(Sketch)
For the direction $\subseteq$, let $c_0 c_1\ldots c_n$ be an execution under $\Rightarrow_I$, where $c_n$ is a final configuration. We need to show that the history $\hist_n$ contained in $c_n$ belongs to $\histOf[*]{\prog}$ and that it satisfies $I$. The fact that $\hist_n\in \histOf[*]{\prog}$ is a direct consequence of the fact that $\Rightarrow_I$ is more constrained than $\Rightarrow$. To prove that $\hist_n$ satisfies $I$, let $c_j$ be the latest configuration in the execution that is obtained from $c_{j-1}$ through an application of \textsc{read-extern}. By the definition of this rule, the history $\hist_j$ in $c_j$ satisfies $I$. Since the write-read relation of $\hist_j$ is identical to that of $\hist_n$, any axiom of the form (\ref{eq:axiom}) satisfied by $\hist_j$ is also satisfied by $\hist_n$ (the set of instantiations of $\forall \tr_1$ and $\forall \alpha$ in (\ref{eq:axiom}) that satisfy the left part of the entailment are the same in $\hist_j$ and $\hist_n$). Therefore, $\hist_n$ satisfies $I$, which concludes this part of the proof.

For the reverse, let $\hist=\tup{T, \so, \wro}\in \histOf[*]{\prog}$ that satisfies $I$. Since $\hist$ satisfies $I$, there exists a commit order $\co$ such that $\wro\cup\so\subseteq \co$ and $\tup{h,\co}$ satisfies the axioms defining $I$. We show that there exists an execution $c_0 c_1\ldots c_n$ under $\Rightarrow_I$ where transactions are executed serially in the order defined by $\co$, such that $c_n$ is a final configuration that contains $\hist$. The only difficulty is showing that the \textsc{read-extern} transitions between two configurations $c_j$ and $c_{j+1}$ that add a write-read dependency $(\tr',\rd{\key}{\val})\in\wro$ are enabled even though the transaction log $\tr$ containing $\rd{\key}{\val}$ is ``incomplete'' in the history $\hist_j$ of $c_j$, and $\hist_j$ does not contain transactions committed after $\tr$. This relies on the prefix-closure property in Lemma~\ref{lem:prefix}.
%defined hereafter. Intuitively, the history of $c_j$ is a ``prefix'' of $\hist$ and any axiom satisfied by $\hist$ is also satisfied by all its prefixes.
Let $\co_j$ be the order in which transactions have been executed until $c_j$. Then, $\tup{\hist_j,\co_j}$ is a prefix of $\tup{\hist,\co}$, and $\tup{\hist_j,\co_j}\models I$ because $\tup{\hist,\co}\models I$.
 \end{proof}

