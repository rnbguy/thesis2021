%!TEX root = Thesis.tex
\section{Introduction}
\label{sec:intro}

Data storage is no longer about writing data to a single
disk with a single point of access. Modern applications require not just data
reliability, but also high-throughput concurrent accesses. 
Applications concerning supply chains, banking, etc. use traditional relational databases
for storing and processing data, whereas applications such as social networking
software and e-commerce platforms 
use cloud-based storage systems (such as Azure CosmosDb \cite{cosmosdb}, Amazon DynamoDb
\cite{decandia2007dynamo}, Facebook TAO \cite{facebook-tao}, etc.). We use the term \textit{storage
system} in this paper to refer to any such database system/service.


%As applications have moved from a single-box environment to the cloud, the notion of
%data persistence has also changed. It is no longer about storing data on a
%single disk with a single point of access. Rather, modern applications such as
%social networking software, e-commerce platforms, cloud micro-services, etc. are built using 
%high-scale storage systems, such as Azure CosmosDb \cite{cosmosdb}, Amazon DynamoDb \cite{amazon-dynamodb}, 
%Facebook TAO \cite{facebook-tao}. Applications such as 
 
%These storage systems, commonly offered by most major cloud providers (such as
%Azure CosmosDb \cite{cosmosdb}, Amazon DynamoDb \cite{amazon-dynamodb}, 
%Facebook TAO \cite{facebook-tao}, etc.)
%create and manage multiple replicas of data. Having multiple replicas offers reliability and prevents
%data loss, but it also offers availability and low-latency accesses by allowing
%different clients to connect with different replicas. 

Providing high-throughput processing, unfortunately, comes at an unavoidable cost of weakening 
the guarantees offered to users.
Concurrently-connected clients may end up observing different views of the same data. 
These ``anomalies'' can be prevented by using a strong \textit{isolation level} 
such as \textit{serializability}, which essentially offers a single view of the
data. However, serializability requires expensive synchronization and incurs a high performance cost. As a
consequence, most storage systems use weaker isolation levels, such as 
{\it Causal Consistency}~\cite{DBLP:journals/cacm/Lamport78,DBLP:conf/sosp/LloydFKA11,antidote-white-paper},
{\it Snapshot Isolation}~\cite{DBLP:conf/sigmod/BerensonBGMOO95}, {\it Read
Committed}~\cite{DBLP:conf/sigmod/BerensonBGMOO95}, etc. for better performance.
In a recent survey of
database administrators \cite{survey}, 86\% of the participants responded that
most or all of the transactions in their databases execute at read committed isolation level.

\begin{figure}
	\begin{minipage}{4.2cm}
		\begin{lstlisting}[basicstyle=\ttfamily\footnotesize,escapeinside={(*}{*)},language=MyLang]
// Append item to cart
AddItem(item i, userId) {
  Begin()
  key = "cart:" + userId
  cart = read(key)
  cart.append(i)
  write(key, cart)
  Commit()
}
		\end{lstlisting}
	\end{minipage}
	\hspace{-5mm}
	\begin{minipage}{4.2cm}
		\begin{lstlisting}[xleftmargin=4mm,basicstyle=\ttfamily\footnotesize,escapeinside={(*}{*)},language=MyLang]
// Fetch cart and delete item
DeleteItem(item i, userId) {
  Begin()
  key = "cart:" + userId
  cart = read(key)
  cart.remove(i)
  write(key, cart)
  Commit()
}
		\end{lstlisting}
	\end{minipage}
	
\vspace{-6mm}	
  \resizebox{8.5cm}{!}{
   \begin{tikzpicture}[->,>=stealth',shorten >=1pt,auto,node distance=4cm,
     semithick, transform shape]
    \node (s11l) at (1.15, 2.1) {\textbf{Initial state}};
    \node[draw, rounded corners=2mm] (t0) at (2.05, 1.5) {\begin{tabular}{l} $\wrt{\texttt{cart:}u}{\{..\, I\, ..\}}$ \end{tabular}};
    \node[draw, rounded corners=2mm, minimum width=3.6cm, minimum height=1.3cm] (s1) at (0, -0.1) {};
    \node[style={inner sep=0,outer sep=0}] (s11) at (0, 0.3) {\begin{tabular}{l} $\rd{\texttt{cart:}u}{\{..\, I\, ..\}}$\end{tabular}};
    \node[style={inner sep=0,outer sep=0}] (s12) at (0, -0.5) {\begin{tabular}{l} $\wrt{\texttt{cart:}u}{\{..\, I,I\, ..\}}$ \end{tabular}};
    \node (s11l) at (-1, 0.8) {\textbf{AddItem}};
    \node[draw, rounded corners=2mm, minimum width=3.6cm, minimum height=1.3cm] (s2) at (4.1, -0.1) {};
    \node[style={inner sep=0,outer sep=0}] (s21) at (4.1, 0.3) {\begin{tabular}{l} $\rd{\texttt{cart:}u}{\{..\, I\, ..\}}$ \end{tabular}};
    \node[style={inner sep=0,outer sep=0}] (s22) at (4.1, -0.5) {\begin{tabular}{l} $\wrt{\texttt{cart:}u}{\{..\, ..\}}$ \end{tabular}};
    \node (s11l) at (4.9, 0.8) {\textbf{DeleteItem}};
    \node[draw, rounded corners=2mm] (r1) at (8.3, 0) {\begin{tabular}{l} $\rd{\texttt{cart:}u}{\{..\, ..\}}$ \end{tabular}};
    \node[draw, rounded corners=2mm] (r2) at (8.3, -1.3) {\begin{tabular}{l} $\rd{\texttt{cart:}u}{\{..\, I, I\, .\}}$ \end{tabular}};
    \path (s11) edge[left] node {$\po$} (s12);
    \path (s21) edge[left] node {$\po$} (s22);
    \path (t0) edge[left] node {$\wro$} (s1);
    \path (t0) edge[right] node {$\wro$} (s2);
    \path (r1) edge[left] node {$\so$} (r2);
    \path (s2) edge[above] node {$\wro$} (r1);
    \path (s1) edge[below,bend right=11] node {$\wro$} (r2);
%    \path (t0) edge[red, right, bend left=20] node[pos=0.4,xshift=-1] {$\wro$} (s11);
%    \path (t0) edge[red, left, bend right=20] node[pos=0.9,xshift=-1] {$\wro$} (s12);
   \end{tikzpicture}  
  }

%  \begin{lstlisting}[xleftmargin=4mm,basicstyle=\ttfamily\footnotesize,escapeinside={(*}{*)},language=MyLang,morekeywords={Test,GetCart}]
%Test: 
%{ AddItem(I, u); GetCart(u); GetCart(u) } || DeleteItem(I, u)
%		\end{lstlisting}
\vspace{-2mm}
	\caption{A simple shopping cart service.}
	\label{fig:motiv}
\vspace{-3mm}
\end{figure}

A weaker isolation level allows for more possible behaviors than stronger
isolation levels. It is up to the developers then to ensure that their
application can tolerate this larger set of behaviors. Unfortunately, weak
isolation levels are hard to understand or reason about
\cite{DBLP:conf/popl/BrutschyD0V17,adya-thesis} and resulting application bugs
can cause loss of business \cite{acidrain}.
Consider a simple shopping cart application that stores a per-client shopping
cart in a key-value store (\textit{key} is the client ID and \textit{value} is a
multi-set of items). \figref{motiv} shows procedures for adding an item to the cart
(\texttt{AddItem}) and deleting \textit{all} instances of an item from the cart
(\texttt{DeleteItem}). Each procedure executes in a transaction, represented by
the calls to \texttt{Begin} and \texttt{Commit}. Suppose that initially, a user $u$ has 
a single instance of item $I$ in their cart.
Then the user connects to the application via two different
sessions (for instance, via two browser windows), adds $I$ in one session
(\texttt{AddItem($I$, $u$)}) and deletes $I$ in the other session
(\texttt{DeleteItem($I$, $u$)}). With serializability, the cart can either be
left in the state $\{ I \}$ (delete happened first, followed by the add) or $\emptyset$ (delete
happened second). However, with causal consistency (or read committed), it is possible that with two
sequential reads of the shopping cart, the cart is empty in the first read
(signaling that the delete has succeeded), but there are \textit{two} instances of $I$ 
in the second read! Such anomalies, of deleted items reappearing, have been
noted in previous work \cite{decandia2007dynamo}. 

\paragraph{Testing storage-based applications}
This paper addresses the problem of \textit{testing} code for correctness
against weak behaviors: a developer should be able to write a test that runs
their application and then asserts for correct behavior. 
The main difficulty today is getting coverage of weak behaviors during
the test. If one runs the test
against the actual production storage system, it is very likely to only result in
serializable behaviors because of their optimized implementation. For
instance, only 0.0004\% of all reads performed on Facebook's TAO storage system 
were not serializable \cite{facebook-consistency}. 
Emulators, offered by cloud providers for local development, on the other hand, do not support weaker
isolation levels at all \cite{cosmosdb-local}. Another option, possible when
the storage system is available open-source, is to set it up with a 
tool like Jepsen~\cite{jepsen} to inject noise (bring down replicas or
delay packets on the network). 
This approach is unable to provide good coverage at the level of client operations
\cite{clotho} (\sectref{oltp}). Another line of work has focussed on finding
anomalies by identifying non-serializable behavior (\sectref{related}). Anomalies, however, do not
always correspond to bugs \cite{DBLP:conf/pldi/BrutschyD0V18,isodiff}; they may
either not be important (e.g., gather statistics) or may already be handled in
the application (e.g., checking and deleting duplicate items).

%Prior work on this problem has largely focussed on
%formal verification techniques: to establish correctness of code against a specification 
%of a particular isolation level \cite{clotho,maryam} (\akash{others?}). Verification requires
%statically analyzing application code; with such an approach, in addition to
%scalability problems, it is often difficult to support
%various programming styles, libraries, frameworks and languages. 
%For these reasons, testing is still the more widely adopted engineering practice. 
%Our goal is support testing of any application, with little to no modifications.
%We defer more details to the related work section.


%
%There 
%is informal documentation available \cite{cosmosdb-consistency} as well as
%formal specifications \akash{cite CosmosDb-TLA}, but none of it is immediately 
%actionable for a developer. 


%Modern applications such as social networking, ecommerce, etc., use
%highly-available low-latency geo-replicated storage systems~\cite{cosmosdb} to
%achieve high performance and scalability.
%These storage systems must replicate data for persistence, and then allow
%clients to connect with different replicas for availability on failures and for
%low latency.
%The replicas communicate updates to each other in the background using message
%passing.
%However, unfortunately, maintaining strong consistency across these replicas
%requires global synchronization which incurs high performance overheads.
%Moreover, as stated by the Consistency, Availability, and Partition-tolerance
%(CAP) theorem~\cite{cap-theorem},
%it is not possible for such storage systems to remain available and
%simultaneously guarantee consistency under network partitions (which are
%unavoidable).
%Hence, to provide high availability and low latency, many distributed data
%stores provide only weak consistency guarantees, formally defined as different
%consistency models: {\it Causal Consistency}~\cite{constantin-causal,DBLP:journals/cacm/Lamport78},
%{\it Snapshot Isolation}~\cite{DBLP:conf/sigmod/BerensonBGMOO95}, and {\it Read
%Committed}~\cite{DBLP:conf/sigmod/BerensonBGMOO95}, etc.
%This current scenario is also showcased in a recent Database Admin
%Survey~\cite{survey} where more than 73\% participants responded that all the
%transactions in their databases execute at read committed consistency level.

%THE NEXT PARAGRAPH SHOULD BE ONLY ABOUT: PROGRAMMING ON TOP OF WEAK ISOLATION IS HARD. 

%The weak isolation semantics of these consistency models permit various
%anomalies that violate data consistency; for example, lost updates,
%non-repeatable reads etc. 
%(DIRTY READS IS NOT AN ANOMALY ABOVE READ COMMITTED).
%Such anomalies often lead to undesirable executions in client applications and manifest in the form of invariant violations (assertion violations).
%For example, consider an online store with a shopping cart
%service~\ref{fig:motiv}.
%If a user is accessing the cart from multiple clients, and deletes an item from
%one client. 
%Under weak consistency, not only that delete operation can take some time to be
%visible through another client, but even after viewing deletion, the item could
%appear again in the cart \cite{decandia2007dynamo}.
%%\textcolor{blue}{Shopping cart example showing assertion violation on weak consistency.}
%To prevent them, developers should be aware of such anomalies and use explicit
%synchronization at appropriate program points in their applications. 
%(TODO: LOCKING IS TOO SPECIFIC, USE SYNCHRONIZATION INSTEAD).
%This requirement makes application development extremely challenging, because
%such weak isolation semantics are hard to understand and
%reason~\cite{DBLP:conf/popl/BrutschyD0V17,adya-thesis}, compared to the simple case of
%serializability where one can argue about one transaction at a time. Further,
%often these consistency levels are informally explained with low-level
%implementation details, leading to poor understanding.
%For example, Cosmos DB~\cite{cosmosdb} defines five levels of consistency with
%only rough guidelines on which one to pick~\cite{cosmosdb-consistency}.
%(THE FLOW IS A LITTLE BIT AKWARD: WEAK ISOLATION IS HARD TO UNDERSTAND COMPARED TO SERIALIZABILITY; MENTION THE Cosmos DB STUFF AS SUPPORT => INSERTING THE RIGHT SYNCHRONIZATION IS HARD => APPLICATION DEVELOPMENT IS HARD. WHY IS TESTING HARD BECAUSE WEAK ISOLATION IS COMPLICATED ?) 

We present MonkeyDB, a mock in-memory storage system meant for testing
correctness of storage-backed applications. 
MonkeyDB supports 
common APIs for accessing data (key-value updates, as well as SQL queries),
making it an easy substitute for an actual storage system. MonkeyDB
can be configured with one of several isolation levels. 
%Currently,
%MonkeyDB supports Serializability, Causal Consistency as well as Read
%Committed. (Addition of other isolation levels is easy.)
On a read operation, MonkeyDB computes the set of all possible return values
allowed under the chosen isolation level, and randomly returns one of them. The
developer can then simply execute their test multiple times to get coverage of
possible weak behaviors. For the program in \figref{motiv}, if we write a test
asserting that two sequential reads cannot return empty-cart followed by $\{I,
I\}$, then it takes only 20 runs of the test (on average) to fail the
assert. In contrast, the test does not fail when using MySQL with read committed, 
even after 100k runs. 
%MonkeyDB can work with any application with little to no modifications:
%a developer simply needs to link their tests to MonkeyDB, instead of the
%production storage system.

\paragraph{Design of MonkeyDB}
MonkeyDB does not rely on stress generation, fault
injection, or data replication. 
Rather, it works directly with a formalization of
the given isolation level in order to compute allowed return values. 


%MonkeyDB makes it straightforward
%to unit test any storage-backed application. Testing, as opposed to formal
%verification, is still the more widely adopted engineering practice. 

%Given these concerns, we propose our idea {\emph \tool{}}, 
% an in-memory storage system meant for testing. 
%It offers the same interface as most databases (both SQL and Key-Value stores). 
%Internally, it uses a generative model of various (configurable) consistency levels. 
%On read operations, it can systematically pick from the set of all values valid under the given consistency level. 
%%a database model which systematically explores possible behaviors allowed by the used consistency model 
%% and allows developers to check whether their application invariants hold with respect to these possible executions, providing high coverage.
%\tool{} can be used for exhaustive testing (by enumerating all possible read values) or randomized testing that provides much better coverage than otherwise (THIS STATEMENT IS STRONG AND NEEDS SUPPORT).
%%{\tool{}} works as an operational model 
%%and provides the interface of an actual database to any client application. 
%Our design choice addresses above-mentioned concerns and is advantageous in several ways.
%First, it does not need to access the internals of any real database.
%Further, it requires no modifications to the client application under test.
%Furthermore, we can use the same model for a number of databases which leads to effortless testing of client applications against various levels of consistency that are provided by different databases (WHAT DO YOU MEAN BY "SAME MODEL"?) 
%Our design makes \tool{} readily support different modern software architectures such as microservices~\cite{microservice,eshop} where varied databases can be plugged-in and each microservice connects to its own database having a specific consistency model.
%%Our effort is building something that any developer can use by themselves.
%%The key features of \tool{} is:
%% (1) works on real applications directly and requires no modifications to the system under test,
%% (2) expects no expertise from developers because its integration and use is same as that of a real database,
%% (3) provides good coverage,
%% (4) does not report false positives.


%\constantin{FROM THIS POINT ON, WE NEED A "DEEPER" PRESENTATION OF THE THEORY
%  USED IN THE IMPLEMENTATION. STARTING FROM AXIOMATIC SEMANTICS FOR ISOLATION
%  LEVELS IN KEY-VALUE STORES, OPERATIONAL SEMANTICS (STRESS THE SERIAL
%EXECUTION), COMPILING SQL TO KEY-VALUE STORE API. }

The theory behind MonkeyDB builds on the axiomatic definitions of isolation
levels introduced by Biswas et al. \cite{DBLP:journals/pacmpl/BiswasE19}. These
definitions use logical constraints (called \emph{axioms}) to characterize the
set of executions of a key-value store that conform to a particular isolation
level (we discuss SQL queries later). These constraints refer to a specific set of
relations between events/transactions in an execution that describe control-flow
or data-flow dependencies: a program order $\po$ between events in the same
transaction, a session order $\so$ between transactions in the same session\footnote{A
session is a sequential interface to the storage system. It corresponds to what
is also called a connection.}, and a write-read $\wro$ (read-from) relation that
associates each read event with a transaction that writes the value returned by
the read. These relations along with the events (also called, operations) in an
execution are called a \emph{history}. The history corresponding to the 
shopping cart anomaly explained above is given on the bottom of Figure~\ref{fig:motiv}.
Read operations include the read value, and boxes group events from the same transaction.
%The initial value of the key is supposed to be written in a fictitious transaction. 
A history describes only the
interaction with the key-value store, omitting application side events (e.g., computing
the value to be written to a key). 

MonkeyDB implements a \emph{centralized} operational semantics for key-value stores, which is based on these axiomatic definitions. Transactions are executed \emph{serially}, one after another, the concurrency being simulated during the handling of read events.  
This semantics maintains a history that contains all the past events (from all
transactions/sessions), and write events are simply added to the history. The
value returned by a read event is established based on a non-deterministic
choice of a write-read dependency (concerning this read event) that satisfies
the axioms of the considered isolation level.
%(MonkeyDB resolves any non-determinism in a random fashion). 
Depending on the weakness of the isolation
level, this makes it possible to return values written in arbitrarily ``old''
transactions, and simulate any concurrent behavior. For instance, the history in Figure~\ref{fig:motiv}
can be obtained by executing \texttt{AddItem}, \texttt{DeleteItem}, and then the two reads (serially).
The read in \texttt{DeleteItem} can take its value from the initial state and ``ignore'' the
previously executed \texttt{AddItem}, because the obtained history validates the axioms of 
causal consistency (or read committed). The same happens for the two later reads in the same
session, the first one being able to read from \texttt{DeleteItem} and the second one
from \texttt{AddItem}.

We formally prove that this semantics does indeed simulate any concurrent behavior, by 
showing that it is equivalent to a semantics where transactions are allowed to interleave.
In comparison with concrete implementations, this semantics makes it possible to handle 
a wide range of isolation levels in a uniform way. It only has two sources of
non-determinism: 
the order in which entire transactions are submitted, and the choice of write-read dependencies in read 
events. This enable better coverage of possible behaviors, the penalty in performance not
being an issue in safety testing workloads which are usually small (see our evaluation). 


We also extend our semantics to cover SQL queries as well, by compiling SQL queries down to transactions with multiple key-value reads/writes. A table in a relational database is represented using a set of primary key values (identifying uniquely the set of rows) and a set of keys, one for each cell in the table. The set of primary key values is represented using a set of Boolean key-value pairs that simulate its characteristic function (adding or removing an element corresponds to updating one of these keys to $\btrue$ or $\bfalse$). Then, SQL queries are compiled to read or write accesses to the keys representing a table. For instance, a $\mathtt{SELECT}$ query that retrieves the set of rows in a table that satisfy a $\mathtt{WHERE}$ condition is compiled to (1) reading Boolean keys to identify the primary key values of the rows contained in the table, (2) reading keys that represent columns used in the $\mathtt{WHERE}$ condition, and (3) reading all the keys that represent cells in a row satisfying the $\mathtt{WHERE}$ condition. This rewriting contains the minimal set of accesses to the cells of a table that are needed to ensure the conventional specification of SQL.
It makes it possible to ``export'' formalizations of key-value store isolation levels to SQL transactions.

%This paper first presents an axiomatic semantics for various isolation levels 
%in Key-Value stores that allows one to reason about the set of valid behaviors
%under a given isolation level. We follow this by a non-deterministic operation
%semantics, equivalent to the axiomatic semantics, where
%each operation is cooperatively scheduled to execute one at a time (serially).
%The operational semantics maintains a history of the read-write operations,
%including a \textit{read-from} relationship that matches reads to previous
%writes. On the submission of a new operation, this history is extended in
%accordance with this operational semantics. MonkeyDB implements the operational
%semantics, while resolving any non-determinism in a random fashion. 
%We also extend our semantics to cover SQL queries as well, by compiling SQL
%queries down to transactions with multiple key-value updates.

\paragraph{Contributions}
%We implemented MonkeyDB to support an interface consistent with key-value
%stores and databases, which allows us to directly run real applications unmodified. 
%We evaluated MonkeyDB on a series of micro-benchmarks, inspired from real
%applications, as well as the well-known OLTPBench \cite{difallah2013oltp}
%that is used for evaluating
%databases on OLTP workloads. 

This paper makes the following contributions:
\begin{itemize}
\item We define an operational semantics for key-value stores under various
  isolation levels, which simulates all concurrent behaviors with executions
  where transactions execute serially (\sectref{op-kv}) and which is based 
  on the axiomatic definitions in~\cite{DBLP:journals/pacmpl/BiswasE19} (and outlined in \S\ref{sec:ax-kv}),
\item We broaden the scope of the key-value store semantics to SQL transactions
  using a compiler that rewrites SQL queries to key-value accesses (\sectref{SQL-to-KV}),
\item The operational semantics and the SQL compiler are implemented in a tool
  called MonkeyDB (\sectref{impl}). It randomly resolves possible choices to provide coverage
  of weak behaviors. It supports both a key-value interface as well as SQL,
  making it readily compatible with any storage-backed application.
\item We present an evaluation of MonkeyDB on several applications, showcasing
its superior coverage of weak behaviors as well as bug-finding abilities
(\sectref{micro}, \sectref{oltp}).\footnote{Source code of our benchmarks is available
as supplementary material.}
\end{itemize}



 


 



